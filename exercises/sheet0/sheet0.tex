\documentclass[11pt]{amsart}

\usepackage{mathptmx}
\usepackage{a4wide}
\usepackage{paralist}
\usepackage{url}
\usepackage{bbm}
\usepackage{fancyvrb}
\usepackage{xcolor}
\usepackage{nopageno}

\newcommand{\cA}{\mathcal{A}}
\newcommand{\cS}{\mathcal{S}}
\DeclareMathOperator{\conv}{conv}
\DeclareMathOperator{\New}{New}
\DeclareMathOperator{\area}{area}
\newcommand{\RR}{\mathbbm{R}}
\newcommand{\CC}{\mathbbm{C}}

\newcommand{\defn}[1]{{\color{blue}#1}}
\newcommand{\alert}[1]{\textbf{\color{red}#1}}

\DefineShortVerb{\;}
\renewcommand{\FancyVerbFormatCom}{\color{green!30!black}}

\begin{document}
\begin{center}
\textbf{\sffamily
   Discrete and Algorithmic Geometry }

\medskip
   Julian Pfeifle,
   UPC, 2019
\end{center}

\bigskip

\begin{center}
  \textbf{\sffamily Sheet 0.}

  \bigskip
  \bigskip
  \bigskip
  Due on Tuesday, November 12, 2019.

\end{center}

\bigskip
\bigskip
\bigskip

\begin{enumerate}
  \setcounter{enumi}{-5}
\item
  Learn about the versioning software ;git;, and practice until you become comfortable using it.
  Then check out the repository of this course using
  \begin{center}
    ;git clone git@gitlab.mat-apl.upc.edu:julian.pfeifle/2019-dag-upc; .
  \end{center}

  \medskip
\item
  Create a new branch ;your-name-cv; in the repository, and edit the file ;participants.tex; to include a short cv and some information about your mathematical interests.
  Then ;commmit; and ;push; your changes. When you feel your work is done, log in and create a pull request at ;gitlab.mat-apl.upc.edu; so that all the different stories may be merged.
  
  \medskip
\item
  Learn about public key cryptography and the use, advantages and disadvantages of the software ;gpg;.

  \medskip
\item
  Organize into teams of 2--3 people to work on the exercises, and
  edit  ;participants.tex; to reflect this.
  As always, ;commit; and ;push; your changes, and create a pull request at ;gitlab.mat-apl.upc.edu;.

  \medskip
\item
  Read up on \alert{two} programming languages of your choice that your team will use in this course.
  One of these should be a scripting language for rapid iteration, the other a compiled language for efficiency.
  If you have never programmed before, a good choice for a scripted language is ;python;/;sage;, and a good choice for a compiled language is ;julia;.
  If you already know some languages, take the opportunity to learn a new one! Some suggestions are ;c++;, ;perl;/;raku;, ;rust;, ;haskell;, ;ocaml;, ;ruby;, ;lisp;/;scheme;
  (I myself have not used all of these).

  \smallskip
  When developing your programs, bear in mind that since we all share the same repository,
  your code is likely to be read by the members of other teams who would like to compare the language they're using to yours.
  Therefore, please take the opportunity to document your code very well, so as to make it as easy for them as possible to understand what you're doing!
  In return, you'll be rewarded by awesomely documented code in languages you didn't have time to learn, but where you know exactly what problems it is solving.

  \smallskip
  Use the facilities that ;git; offers to synchronize and collaborate on your code across devices and operating systems.
\end{enumerate}
\end{document}

\end{document}
