\documentclass[11pt]{amsart}

\usepackage{mathptmx}
\usepackage{a4wide}
\usepackage{paralist}
\usepackage{url}
\usepackage{bbm}
\usepackage{fancyvrb}
\usepackage{hyperref}
\usepackage{xcolor}
\usepackage{tikz}
\usetikzlibrary{arrows,positioning}

\newcommand{\cA}{\mathcal{A}}
\newcommand{\cT}{\mathcal{T}}
\newcommand{\cS}{\mathcal{S}}
\DeclareMathOperator{\conv}{conv}
\DeclareMathOperator{\New}{New}
\DeclareMathOperator{\area}{area}
\DeclareMathOperator{\del}{del}
\DeclareMathOperator{\link}{link}
\DeclareMathOperator{\thestar}{star}
\DeclareMathOperator{\GL}{GL}
\DeclareMathOperator{\corank}{corank}
\DeclareMathOperator{\vol}{vol}
\DeclareMathOperator{\aff}{aff}

\newcommand{\NN}{\mathbbm{N}}
\newcommand{\RR}{\mathbbm{R}}
\newcommand{\CC}{\mathbbm{C}}
\newcommand{\kk}{\mathbbm{k}}
\newcommand{\cI}{\mathcal{I}}

\newcommand{\aaa}{\mathbf{a}}

\newcommand{\defn}[1]{{\color{blue}#1}}
\newcommand{\alert}[1]{\textbf{\color{red}#1}}

\renewcommand{\FancyVerbFormatCom}{\color{green!30!black}}

\newtheorem*{thmstar}{Theorem}

\begin{document}
\DefineShortVerb{\·}

\begin{center}
\textbf{\sffamily
   Discrete and Algorithmic Geometry }

\medskip
   Julian Pfeifle,
   UPC, 2019
\end{center}

\medskip

\begin{center}
  \textbf{\sffamily Sheet 2}

  \bigskip
Due on Thursday, December 19, 2019
\end{center}

\bigskip

You should work in your teams and complete half of the exercises on this sheet.
Submissions work in the same way as for Sheet 1.


\bigskip
\begin{enumerate}

\item
  Let $\cA=(a_1,\dots,a_{d+2})\subset\RR^d$ be a full-dimensional point configuration of $d+2$ points,
  so that the associated vector configuration that one gets by homogenizing coordinates has rank~$d+1$, and therefore $\corank\cA=1$.
  Let $\lambda\in\RR^{d+2}$ be the vector of coefficients of the unique affine dependency $J=(J_+,J_-)$ among the members of~$\cA$.
  \begin{enumerate}
  \item Show using Cramer's rule that, up to a global sign,
    \[
      \lambda_i
      \ = \
      (\text{volume of the simplex $\cA\setminus a_i$})\cdot
      \begin{cases}
        +1
        &
        \text{if } i \in J_+
        \\
        -1
        &
        \text{if } i \in J_-        
      \end{cases}
    \]

  \item 
    Let $T_\pm=\{\cA\setminus a_j:j\in J_\pm\}$ be the two unique triangulations of~$\cA$,
    and let
    \[
      \phi_{\cA}(T_\pm)
      \ = \
      \sum_{j=1}^{d+2}\vol\thestar_{T_\pm}(a_j)\,e_j
    \]
    be their corresponding GKZ-vectors. Show that
    \[
      \big(\phi_{\cA}(T_+)\big)_i
      \ = \
      \begin{cases}
        \vol\,\conv\cA
        &
        \text{if } i\in J_-,
        \\
        \vol\,\conv\cA - \vol\,\conv(\cA\setminus a_i)
        &
        \text{if } i\in J_+.        
      \end{cases}
    \]

  \item
    What is the relationship between $\phi_\cA(T_+)$, $\phi_\cA(T_-)$, and $\lambda$?
  \end{enumerate}

  \bigskip

\item
  Consider a full-dimensional point configuration $\cA\subset\RR^d$.
  Show that the affine hull of the secondary polytope $\Sigma$-poly$(\cA)$ is given by
  \begin{align*}
    \sum_{\aaa\in\cA} x_\aaa \
    &= \
      d(d+1)\vol\,\conv\cA,
    \\
    \sum_{\aaa\in\cA} x_\aaa\,\aaa \
    &= \
      \big((d+1)\vol\,\conv\cA\big)\mathbf{c}_\cA,
  \end{align*}
  where $\mathbf{c}_\cA$ denotes the centroid of $\conv\cA$.

  \bigskip

\item
  Show that every configuration $\cA$ of $d+4$ points in dimension~$d$
  has at most $O(d^4)$ regular triangulations:
  \begin{enumerate}
  \item
    Think of their chamber complexes as cell decompositions of a $2$-sphere.

  \item
    Bound the number of vertices in these chamber complexes by $O(d^4)$,
    showing that every vertex corresponds
    (not uniquely, but that is not a problem)
    to a pair of circuits of~$\cA$.

  \item
    Using Euler's formula, show that any (polyhedral) cell
    decomposition of a $2$-sphere has about as many vertices as it has
    $2$-cells. More precisely, these two numbers are within a factor of~$2$
     of each other.
   \end{enumerate}

   \bigskip

 \item
   Show that the cyclic polytope $C_{4k}(4k+4)$ has at least $\frac14\,2^k$ many
   non-regular triangulations:
   \begin{enumerate}
   \item
     Consider first the coordinatization having a cyclically symmetric affine
     Gale diagram, in which $k$~lines meet in a zero-dimensional chamber
     at the center.

   \item
     Show that by perturbing the coordinates, each of these $k$~lines can be
     moved towards one or the other side of the center.
     Since the cyclic configuration is in general position,
     these perturbations produce different coordinatizations of the same oriented matroid,
     hence give the same polytope.

   \item
     Show that for all these perturbations the center of the Gale diagram
     corresponds to distinct triangulations. For this, use the description
     of a chamber as an intersection of the ``dual simplicial cones'' of the
     corresponding triangulation.
   \end{enumerate}
  \bigskip
  
\item
  Let $\cT=\{\sigma_1,\dots,\sigma_m\}$ be a triangulation of a full-dimensional point configuration $\cA=(\aaa_1,\dots,\aaa_n)\subset\RR^d$ of $n$~points,
  where we consider the $\sigma_i\in\binom{[n]}{d+1}$ to be index sets.
  Let $R^{\text{int}}$ be the set of \defn{interior ridges},
  defined to be intersections $\rho = \rho_{ij} = \sigma_i\cap\sigma_j$ of two facets of~$\cT$
  such that the affine span of the points of~$\cA$ indexed by~$\rho$ has dimension~$d-1$.
  (In the triangulations of~Figure~\ref{fig:triangs1}, they are the interior edges.)

  \begin{figure}[htbp]
    \centering
    \begin{tikzpicture}[scale=.4]
      \coordinate (1) at (0,0);
      \coordinate (2) at (12,-1);
      \coordinate (3) at (6,9);
      \coordinate (4) at (3,1.5);
      \coordinate (5) at (9,1.5);
      \coordinate (6) at (6,6);

      \draw[very thick] (1)--(2)--(3)--(1)--(4)--(5)--(6)--(4);
      \draw[very thick] (2)--(5);
      \draw[very thick] (3)--(6);

      \draw[very thick] (1)--(5);
      \draw[very thick] (2)--(6);
      \draw[very thick] (3)--(4);
    \end{tikzpicture}
    \qquad
    \begin{tikzpicture}[scale=.4]
      \coordinate (1) at (0,0);
      \coordinate (2) at (12,-1);
      \coordinate (3) at (6,9);
      \coordinate (4) at (3,1.5);
      \coordinate (5) at (9,1.5);
      \coordinate (6) at (6,6);

      \draw[very thick] (1)--(2)--(3)--(1)--(4)--(5)--(6)--(4);
      \draw[very thick] (2)--(5);
      \draw[very thick] (3)--(6);

      \draw[very thick] (2)--(4);
      \draw[very thick] (3)--(5);
      \draw[very thick] (1)--(6);
    \end{tikzpicture}    
    \caption{Two triangulations.
    View the source code for the coordinates of the points.}
    \label{fig:triangs1}
  \end{figure}

  \begin{enumerate}
  \item
    For a vector $\omega\in\RR^n$, lift the points in $\cA$ to heights~$\omega$, so that
    $\cA^\omega=\big(\binom{\aaa_1}{\omega_1},\dots,\binom{\aaa_n}{\omega_n}\big)$.
    For each interior ridge $\rho=\sigma_i\cap\sigma_j\in R^{\text{int}}$,
    formulate the \defn{folding condition} that expresses that $\rho$ indexes a face of the lower convex hull of~$\cA^\omega$,
    in terms of the coordinates of the~$\aaa_i$ and~$\omega$.
    Your folding condition should be an inequality that is linear in each height~$\omega_i$.

  \item
    Write code that takes the coordinates of the $\aaa_i$ and the facets $\sigma_i$ of a triangulation as input,
    and outputs the set of folding conditions in a text file in
    \href{https://www.gurobi.com/documentation/8.1/refman/lp_format.html}{LP file format}.

  \item
    Download a linear programming software such as
    \href{https://www.gurobi.com/academia/academic-program-and-licenses/}{\texttt{gurobi}},
    \href{https://www.ibm.com/analytics/cplex-optimizer}{\texttt{cplex}} or
    \href{https://scip.zib.de/}{\texttt{scip}}/\href{https://soplex.zib.de/}{\texttt{soplex}}
    and check explicitly whether there exists a choice of heights $\omega$ that induces each of the triangulations of Figure~\ref{fig:triangs1}.

  \item
    Using this code,
    check that the triangulation of the $4$-dimensional cube from~\cite{deLoera96}
    given by the files \texttt{4-cube.vertices} and \texttt{4-cube.triangulation}
    is non-regular, i.e., it does not come from a lifting to~$\RR^5$.
    If you like, download and play with \href{http://www.rambau.wm.uni-bayreuth.de/TOPCOM}{TOPCOM}.
  \end{enumerate}

  \bigskip

\item
  The \defn{permutohedron} $\Pi^{n-1}$ is the convex hull of all $n!$ permutations of the vector $(1,2,\dots,n)$.
  For $n\ge3$,
  show that $\dim\Pi^{n-1}=n-1$,
  that $f_{n-2}(\Pi^{n-1}) = 2^n-2$,
  and $f_1(\Pi^{n-1}) = \tfrac12 (n-1)n!$.
  If you like, show that $f_k(\Pi^{n-1}) = k!\{{n\atop k}\} = \sum_{i=0}^k(-1)^i\binom{k}{i}(k-i)^n$.
\end{enumerate}

\bibliographystyle{plain}
\bibliography{sheet2}

\end{document}
