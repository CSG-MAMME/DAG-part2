\documentclass[11pt]{amsart}

\usepackage{mathptmx}
\usepackage{a4wide}
\usepackage{paralist}
\usepackage{url}
\usepackage{bbm}
\usepackage{fancyvrb}
\usepackage{xcolor}

\newcommand{\cA}{\mathcal{A}}
\newcommand{\cS}{\mathcal{S}}
\DeclareMathOperator{\conv}{conv}
\DeclareMathOperator{\New}{New}
\DeclareMathOperator{\area}{area}
\newcommand{\RR}{\mathbbm{R}}
\newcommand{\CC}{\mathbbm{C}}

\newcommand{\defn}[1]{{\color{blue}#1}}
\newcommand{\alert}[1]{\textbf{\color{red}#1}}

\DefineShortVerb{\;}
\renewcommand{\FancyVerbFormatCom}{\color{green!30!black}}

\begin{document}
\begin{center}
\textbf{\sffamily
   Discrete and Algorithmic Geometry }

\medskip
   Julian Pfeifle,
   UPC, 2019
\end{center}

\bigskip

\begin{center}
  \textbf{\sffamily Sheet 1}

  \bigskip\bigskip
Due on Tuesday, November 19, 2019
\end{center}

\bigskip


\begin{quote}\small
  To submit your solutions to these exercises,
  \begin{itemize}[$\quad\triangleright$]
  \item create a new branch ;your-awesome-team-name-sheet-1;,
  \item create a subdirectory ;exercises/sheet1/your-awesome-team-name/;,
  \item and put your solutions to the exercises into a ;.pdf; file into that directory.
  \item Now encrypt this ;.pdf; using ;julian.pfeifle@upc.edu.public.gpg.key;, and
  \item ;add;, ;commit; and ;push; \alert{only this encrypted pdf, not the original} ;.tex;
  \item and create a pull request.
  \end{itemize}
  You will be graded collectively on these exercises, and individually in the final exam.

  \alert{Exercises not submitted via this mechanism will not be graded.}
\end{quote}

\bigskip

Let $([n],\mathcal I)$ be a matroid on the ground set $[n]=\{1,2,\dots,n\}$ with independent sets~$\{I:I\in\mathcal I\}$.

\begin{itemize}[$\triangleright$]
\item
  For any proper subset $S\subset[n]$, the \defn{deletion}
  $M\setminus S$ is the matroid on the ground set $[n]\setminus S$ whose
  independent sets are $\{I\subset[n]\setminus S : I\in\mathcal I\}$.

\item
  The \defn{dual matroid} $M^\star$ of~$M$ is the matroid on $[n]$ where $I$~is a basis iff $[n]\setminus I$ is a basis~of~$M$.

\item
  If $S\subset[n]$, then the \defn{contraction} of $M$ with respect to $S$ is $M/S = (M^\star\setminus S)^\star$.

\item
  Let $G$ be a graph whose edges are labeled by $[n]$.
  The bases of the \defn{graphical matroid} $M_G$ are the sets of edges corresponding to spanning trees of~$G$.
\end{itemize}

\bigskip
\begin{enumerate}
\item True or false?
  \begin{enumerate}
  \item This notion of contraction agrees with the notion of
    contraction in graph theory.
  \item $M_{G^\star} = (M_G)^\star$, if $G$ is a planar graph and $G^\star$ its dual planar graph.
  \end{enumerate}
  
  \bigskip
\item Prove that if a matroid $M$ is realizable over a ground field $\mathbbm k$, then the dual matroid~$M^\star$ is also realizable over~$\mathbbm k$.
  {\footnotesize\color{green!30!black} [\emph{Hint.} Suppose that $M$ has rank~$d$ and $n$~elements.
  After a change of basis, $M$~can be realized by the $d\times n$ matrix $A=[I|B]$, where $I$~is the $d\times d$ identity matrix, and $B$~has size $d\times(n-d)$.
  Now find a matrix that realizes~$M^\star$.]
}

\bigskip
\item
  Consider the matroid $M$ realized by the columns of the matrix
  \[
    \begin{bmatrix}
      0 & 1 & 1 & 0 & 1 \\
      1 & 0 & 1 & 0 & 1 \\
      0 & 1 & 1 & 1 & 0
    \end{bmatrix}.
  \]
  Compute a realization of~$M^\star$, and some contractions of~$M$ of your choosing.

\end{enumerate}

\section*{Coding}

\begin{quote}\small
  To submit your solutions to the next two exercises,
  \begin{itemize}[$\quad\triangleright$]
  \item switch to your branch ;your-awesome-team-name-sheet-1;,
  \item create a subdirectory ;exercises/sheet1/your-awesome-team-name/coding;,
  \item and ;add; \alert{all} files you create to your commits, \alert{without encryption}.
  \item Record your results in ;sheet1/results.tex;, and create a pull request.
  \end{itemize}
  This will make it possible to have conversations about your code in your pull requests.
\end{quote}

\bigskip
\begin{enumerate}
  \setcounter{enumi}{3}
\item
  In the programming languages of your choice,
  write code that checks whether the sets of integers contained in the directory ;exercises/sheet1/matroid-or-not; satisfy the matroid basis axioms or not.
  What is the combinatorial complexity of your code?
  What parameters of the data does this combinatorial complexity depend on?

\item
  Write (or use, or search for and download) code that given integers $n\ge k\ge0$ creates all $\binom{n}{k}$~combinations of an $n$-set.
  They form the set of bases of the \defn{uniform matroid of rank~$k$ on $n$~elements}.
  Run your code on various instances of these matroids, and plot the execution time against reasonable parameters.
  Is your conclusion from part (4) borne out?
\end{enumerate}

\end{document}
