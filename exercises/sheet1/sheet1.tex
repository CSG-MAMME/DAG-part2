\documentclass[11pt]{amsart}

\usepackage{mathptmx}
\usepackage{a4wide}
\usepackage{paralist}
\usepackage{url}
\usepackage{bbm}
\usepackage{fancyvrb}
\usepackage{xcolor}

\newcommand{\cA}{\mathcal{A}}
\newcommand{\cS}{\mathcal{S}}
\DeclareMathOperator{\conv}{conv}
\DeclareMathOperator{\New}{New}
\DeclareMathOperator{\area}{area}
\newcommand{\RR}{\mathbbm{R}}
\newcommand{\CC}{\mathbbm{C}}

\newcommand{\defn}[1]{{\color{blue}#1}}
\newcommand{\alert}[1]{\textbf{\color{red}#1}}

\DefineShortVerb{\;}
\renewcommand{\FancyVerbFormatCom}{\color{green!30!black}}

\begin{document}
\begin{center}
\textbf{\sffamily
   Discrete and Algorithmic Geometry }

\medskip
   Julian Pfeifle,
   UPC, 2019
\end{center}

\bigskip

\begin{center}
  \textbf{\sffamily Sheet 0.}

  \bigskip
  \bigskip
  \bigskip
  Due on Tuesday, November 12, 2019.

\end{center}

\bigskip
\bigskip
\bigskip

\begin{enumerate}
  \setcounter{enumi}{-5}
\item
  Learn about the versioning software ;git;, and practice until you become comfortable using it.
  Then check out the repository of this course using
  \begin{center}
    ;git clone git@gitlab.mat-apl.upc.edu:julian.pfeifle/2019-dag-upc; .
  \end{center}

  \medskip
\item
  Create a new branch ;your-name-cv; in the repository, and edit the file ;participants.tex; to include a short cv and some information about your mathematical interests.
  Then ;commmit; and ;push; your changes. When you feel your work is done, log in and create a pull request at ;gitlab.mat-apl.upc.edu; so that all the different stories may be merged.
  
  \medskip
\item
  Learn about public key cryptography and the use, advantages and disadvantages of the software ;gpg;.

  \medskip
\item
  Organize into teams of 2--3 people to work on the exercises, and
  edit  ;participants.tex; to reflect this.
  As always, ;commit; and ;push; your changes, and create a pull request at ;gitlab.mat-apl.upc.edu;.

  \medskip
\item
  Read up on \alert{two} programming languages of your choice that your team will use in this course.
  One of these should be a scripting language for rapid iteration, the other a compiled language for efficiency.
  If you have never programmed before, a good choice for a scripted language is ;python;/;sage;, and a good choice for a compiled language is ;julia;.
  If you already know some languages, take the opportunity to learn a new one! Some suggestions are ;c++;, ;perl;/;raku;, ;rust;, ;haskell;, ;ocaml;, ;ruby;, ;lisp;/;scheme;
  (I myself have not used all of these).

  \smallskip
  When developing your programs, bear in mind that since we all share the same repository,
  your code is likely to be read by the members of other teams who would like to compare the language they're using to yours.
  Therefore, please take the opportunity to document your code very well, so as to make it as easy for them as possible to understand what you're doing!
  In return, you'll be rewarded by awesomely documented code in languages you didn't have time to learn, but where you know exactly what problems it is solving.

  \smallskip
  Use the facilities that ;git; offers to synchronize and collaborate on your code across devices and operating systems.
\end{enumerate}

\newpage
\begin{center}
  \textbf{\sffamily Sheet 1.}
\qquad
Due on Tuesday, November 19, 2019.
\end{center}

\bigskip


\begin{quote}\small
  To submit your solutions to these exercises,
  \begin{itemize}[$\quad\triangleright$]
  \item create a new branch ;your-awesome-team-name-sheet-1;,
  \item create a subdirectory ;exercises/sheet1/your-awesome-team-name/;,
  \item and put your solutions to the exercises into a ;.pdf; file into that directory.
  \item Now encrypt this ;.pdf; using ;julian.pfeifle@upc.edu.public.gpg.key;, and
  \item ;add;, ;commit; and ;push; \alert{only this encrypted pdf, not the original} ;.tex;
  \item and create a pull request.
  \end{itemize}
  You will be graded collectively on these exercises, and individually in the final exam.

  \alert{Exercises not submitted via this mechanism will not be graded.}
\end{quote}

\bigskip

Let $([n],\mathcal I)$ be a matroid on the ground set $[n]=\{1,2,\dots,n\}$ with independent sets~$\{I:I\in\mathcal I\}$.

\begin{itemize}[$\triangleright$]
\item
  For any proper subset $S\subset[n]$, the \defn{deletion}
  $M\setminus S$ is the matroid on the ground set $[n]\setminus S$ whose
  independent sets are $\{I\subset[n]\setminus S : I\in\mathcal I\}$.

\item
  The \defn{dual matroid} $M^\star$ of~$M$ is the matroid on $[n]$ where $I$~is a basis iff $[n]\setminus I$ is a basis~of~$M$.

\item
  If $S\subset[n]$, then the \defn{contraction} of $M$ with respect to $S$ is $M/S = (M^\star\setminus S)^\star$.
  
\end{itemize}

\bigskip
\begin{enumerate}
\item Does this notion of contraction agree with the notion of contraction in graph theory?

  \bigskip
\item Prove that if a matroid $M$ is realizable over a ground field $\mathbbm k$, then the dual matroid~$M^\star$ is also realizable over~$\mathbbm k$.
  {\footnotesize\color{green!30!black} [\emph{Hint.} Suppose that $M$ has rank~$d$ and $n$~elements.
  After a change of basis, $M$~can be realized by the $d\times n$ matrix $A=[I|B]$, where $I$~is the $d\times d$ identity matrix, and $B$~has size $d\times(n-d)$.
  Now find a matrix that realizes~$M^\star$.]
}

\bigskip
\item
  Consider the matroid $M$ realized by the columns of the matrix
  \[
    \begin{bmatrix}
      0 & 1 & 1 & 0 & 1 \\
      1 & 0 & 1 & 0 & 1 \\
      0 & 1 & 1 & 1 & 0
    \end{bmatrix}.
  \]
  Compute a realization of~$M^\star$, and some contractions of~$M$ of your choosing.
\end{enumerate}

\section*{Coding}

\begin{quote}\small
  To submit your solutions to the next two exercises,
  \begin{itemize}[$\quad\triangleright$]
  \item switch to your branch ;your-awesome-team-name-sheet-1;,
  \item create a subdirectory ;exercises/sheet1/your-awesome-team-name/coding;,
  \item and ;add; \alert{all} files you create to your commits, \alert{without encryption}.
  \item Record your results in ;sheet1/results.tex;, and create a pull request.
  \end{itemize}
  This will make it possible to have conversations about your code in your pull requests.
\end{quote}

\bigskip
\begin{enumerate}
  \setcounter{enumi}{3}
\item
  In the programming languages of your choice,
  write code that checks whether the sets of integers contained in the directory ;exercises/sheet1/matroid-or-not; satisfy the matroid basis axioms or not.
  What is the combinatorial complexity of your code?
  What parameters of the data does this combinatorial complexity depend on?

\item
  Write (or use, or search for and download) code that given integers $n\ge k\ge0$ creates all $\binom{n}{k}$~combinations of an $n$-set.
  They form the set of bases of the \defn{uniform matroid of rank~$k$ on $n$~elements}.
  Run your code on various instances of these matroids, and plot the execution time against reasonable parameters.
  Is your conclusion from part (4) borne out?
\end{enumerate}

\end{document}
