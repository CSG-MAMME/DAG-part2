\documentclass[11pt]{amsart}

\usepackage{mathptmx}
\usepackage{a4wide}
\usepackage{paralist}
\usepackage{url}
\usepackage{bbm}
\usepackage{fancyvrb}
\usepackage{hyperref}
\usepackage{xcolor}
\usepackage{tikz}
\usetikzlibrary{arrows,positioning}
\usepackage{kbordermatrix}

\newcommand{\cA}{\mathcal{A}}
\newcommand{\cS}{\mathcal{S}}
\DeclareMathOperator{\conv}{conv}
\DeclareMathOperator{\New}{New}
\DeclareMathOperator{\area}{area}
\DeclareMathOperator{\del}{del}
\DeclareMathOperator{\link}{link}
\DeclareMathOperator{\thestar}{star}
\DeclareMathOperator{\GL}{GL}
\newcommand{\NN}{\mathbbm{N}}
\newcommand{\RR}{\mathbbm{R}}
\newcommand{\CC}{\mathbbm{C}}
\newcommand{\kk}{\mathbbm{k}}
\newcommand{\cI}{\mathcal{I}}

\newcommand{\defn}[1]{{\color{blue}#1}}
\newcommand{\alert}[1]{\textbf{\color{red}#1}}

\renewcommand{\FancyVerbFormatCom}{\color{green!30!black}}

\newtheorem*{thmstar}{Theorem}

\begin{document}
\DefineShortVerb{\·}

\begin{center}
\textbf{\sffamily
   Discrete and Algorithmic Geometry }

\medskip
   Julian Pfeifle,
   UPC, 2019
\end{center}

\medskip

\begin{center}
  \textbf{\sffamily Sheet 1}

  \bigskip
Due on Tuesday, December 10, 2019
\end{center}

\bigskip

You should work in your teams and complete half of the exercises on this sheet.

\bigskip

\begin{quote}\small
  To submit your solutions to these exercises,
  \begin{itemize}[$\quad\triangleright$]
  \item fill out on the \href{https://gitlab.com/julian-upc/2019-dag-upc/wikis/completed-exercises}{course wiki} which exercises your team has completed. 
  \item create a new branch ·your-awesome-team-name-sheet-1·,
  \item create a subdirectory ·exercises/sheet1/your-awesome-team-name/·,
  \item and put your solutions to the exercises into one or more ·.tex· files into that directory.
  \item Now encrypt these ·tex· files using ·julian.pfeifle@upc.edu.public.gpg.key·, 
  \item ·add·, ·commit· and ·push· \alert{only these encrypted files, not the original} ·.tex·,
  \item and create a merge request.
  \end{itemize}
  You will be graded collectively on these exercises, and individually in the final exam.

  \alert{Exercises not submitted via this mechanism will not be graded.}
\end{quote}

\bigskip
\begin{enumerate}
\item Recall the following definitions for a matroid~$M$ on the ground set $[n]=\{1,2,\dots,n\}$ with family of  independent sets $\{I:I\in\mathcal I\}$.

  \smallskip
\begin{itemize}[$\triangleright$]
\item
  For any proper subset $A\subset[n]$, the \defn{deletion}
  $M\setminus A$ is the matroid on the ground set $[n]\setminus A$ whose
  independent sets are $\{I\subset[n]\setminus A : I\in\mathcal I\}$.

\item
  For $a\in E$, the independent sets of the \defn{contraction} $M/a$ are $\big\{I-\{a\}:a\in I\in\mathcal I\big\}$.
  
\item
  The \defn{dual matroid} $M^\star$ of~$M$ is the matroid on $[n]$ where $I$~is a basis iff $[n]\setminus I$ is a basis~of~$M$.
\end{itemize}

\smallskip

Now prove the following statements.

\smallskip

  \begin{enumerate}
  \item Contraction in $M$ as defined in class agrees with the notion of
    contraction in graphs.
  \item Contraction in $M$ as defined in class agrees with $M/S := (M^\star\setminus S)^\star$ for a subset $S\subset[n]$.
  \item $M_{G^\star} = (M_G)^\star$, if $G$ is a planar graph and $G^\star$ its dual planar graph.
  \item 
    Consider the matroid $M$ realized by the columns of the matrix
    \[
      \begin{bmatrix}
        0 & 1 & 1 & 0 & 1 \\
        1 & 0 & 1 & 0 & 1 \\
        0 & 1 & 1 & 1 & 0
      \end{bmatrix}.
    \]
    Compute a realization of~$M^\star$, and some contractions of~$M$ of your choosing.
    Compute the set of circuits and cocircuits of~$M$ and of $M^\star$.
  \end{enumerate}

  \bigskip

\item
  Let $A\in\RR^{d \times n}$ be a matrix of full rank, and let the columns of $B\in\RR^{n\times(n-d)}$ be a basis of the row space of~$A$,
  so that $AB=0$ and the rows of~$B$ are a Gale transform of the columns of~$A$.
  We saw in class that $\mathsf{LinVal}(A)=\mathsf{LinDep}(B)$.
  Show that $\mathsf{LinVal}(B) = \mathsf{LinDep}(A)$.

\bigskip 
  
\item \emph{The greedy algorithm always works for matroids.}
  \begin{enumerate}
  \item
    Show that Kruskal's greedy algorithm always finds a
maximum weight independent set in a matroid $M=(E,\cI)$, regardless of the choice of weight
function $\omega : E\to\RR_{>0}$.
Recall that the \defn{greedy algorithm} starts by setting $I:=\emptyset$,
and next repeatedly chooses $y\in E\setminus I$ with $I\cup\{y\}\in\cI$ and with $\omega(y)$ as large as possible.
It stops if no such $y$~exists.

\item
  Show that that this property characterizes independent sets of matroids
  among all simplicial complexes.
  In other words, given a simplicial complex $\Sigma$ for which the greedy algorithm always works, regardless of the weight function~$\omega$,
  show that $\Sigma = \mathcal I(M)$ for some matroid~$M$.
  \begin{quote}
    \footnotesize\color{green!30!black} 
      Hint: You only need to show that the
      exchange axiom I3 holds. To do this, given $I_1 ,I_2\in\mathcal I$ with $|I_2| = |I_1| + 1 = k + 1$,
      consider the weight function
      \[
        \omega(E) :=
        \begin{cases}
          \frac{k+1}{k+2}
          & \text{for } e\in I_1
          \\
          \frac{k}{k+1}
          & \text{for } e\in I_2\setminus I_1
        \end{cases}.
      \]
      Explain why the greedy algorithm will build up $I_1$ first, and then at the next step,
      will exhibit an element of the form $I_1 \cup \{e\} \in \mathcal I$ with $e \in I_2 - I_1$ . In particular,
      explain why the algorithm will not just stop after having found $I_1$!
    \end{quote}
  \end{enumerate}

  \bigskip

\item
  A famous result by Jack Edmonds states the following:

  \begin{thmstar}[Matroid Intersection Theorem]
    Let $M_1=(E,\cI_1)$ and $M_2=(E,\cI_2)$ be two matroids on the same ground set~$E$, 
    with rank functions
    \[
      r_i:
      2^E\to\NN_{>0},
      \quad
      S\mapsto\max\{|I|:I\in\cI_i,\,I\subseteq S\},
      \qquad
      i=1,2.
    \]
    Then the maximum size of a common independent set in $\cI_1\cap\cI_2$ is equal to
    \[
      \min\{ r_1(S) + r_2(E\setminus S) : S\subseteq E\}.
    \]
  \end{thmstar}

  \begin{enumerate}
  \item
    Let $G=(V,E)$ be a bipartite graph with color classes $V_1,V_2$.
    For $i=1,2$, let $M_i$~be the matroid where $I\subseteq E$ is independent if and only if
    each vertex in~$V_i$ is covered by at most one edge in~$I$.
    Use the Matroid Intersection Theorem to prove
    \begin{thmstar}[K\H onig's Matching Theorem]
      The maximum size of a matching in a bipartite graph equals the minimum size of a vertex cover.
    \end{thmstar}
    (A \defn{vertex cover} of $G$ is a subset of $V$ that intersects each edge.)

    \medskip

  \item
    Let $G=(V,E)$ be a graph whose edges are partitioned into $k$~colors, $E=E_1\cup E_2\cup\cdots\cup E_k$.
    Use the Matroid Intersection Theorem to prove that there exists a \defn{rainbow spanning tree}
    (a spanning tree all of whose edges are colored differently) if and only if
    $G-F$ has at most $t+1$ connected components, for any union~$F$ of $t$~colors, for any $t\ge0$.
    \begin{quote}
    \footnotesize\color{green!30!black} 
    Hint: Apply the Matroid Intersection Theorem to the cycle matroid of $G$ and the partition matroid induced by $E_1,\dots,E_k$.
    Here, the \defn{cycle matroid} on~$G$ is the matroid whose independent sets are the edge sets that form a forest,
    and the \defn{partition matroid} is just the transversal matroid; it is called that because the edges are partitioned into colors.
    Thus, the independent sets in the partition matroid are of the form $\{i_1,\dots,i_s\}$ for $i_j\in E_j$. 
    \end{quote}
  \end{enumerate}
  
  \bigskip
  
\item Prove (a) and (b) of the following implications for matroids, and illustrate them with a well-chosen example.
  Part (c) is optional and depends on your knowledge of algebra.
\DefineShortVerb{|}

\begin{center}
\begin{tikzpicture}\small
  \tikzset{>=stealth}
    \node (algebraic) {Algebraic};
    \node [left=of algebraic] (linear) {Linear}
    edge[->] node[above] {(c)} (algebraic) ;
    \node [above left=of linear] (graphic) {Graphic}
     edge[->] node[auto] {(a)} (linear);
     \node [below left=of linear] (transversal) {Transversal}
     edge[->] node[below right] {(b)} (linear);
  \end{tikzpicture}
\end{center}

\begin{quote}
{\footnotesize\color{green!30!black} 
  Hints.
  \begin{enumerate}[(a)]
    \item Suppose that $G=(V,E)$. In the vector space $\RR^V$ whose standard basis is indexed by the vertices of $G$,
    represent the element $e=(v,w)$ in~$E$ by the vector $e_v-e_w$.
    In other words, show that the linearly independent subsets of these vectors are indexed by forests of edges of~$G$.

    \medskip
  \item
    Given a bipartite graph $G$ with vertex bipartition $E\cup F$, show that you can
represent its transversal matroid as follows. Let $\kk(X):=\kk(x_{e,f}:e\in E,f\in F )$
be a field extension of the field $\kk$ by transcendentals $\{x_{e,f}\}$ indexed by all edges
$\{e,f\}$ of $G$. Then in the vector space $\kk(X)^F$ having standard basis vectors $u_f$
indexed by the vertices $f$ in $F$, represent the element $e\in E$ by the vector
\[
  \sum_{f\in F: \{e,f\}\in G}
  x_{e,f}\, u_f.
\]
In other words, show that the linearly independent subsets of these vectors are
indexed by the subsets of vertices in $E$ that can be matched into $F$ along edges
of $G$.

\medskip
\item
  Given a matroid $M$ of rank $r$ linearly represented by a set of vectors $\{v_1,\dots,v_n\}$
in the vector space~$\kk^r$, represent $M$ algebraically by elements of the rational
function field $\kk(x_1,\dots, x_r)$ as follows.
If $v_i$ has coordinates $(v_{i1},\dots,v_{ir})$ with respect to the standard basis for $\kk^r$,
then represent~$v_i$ by $f_i := \sum_{j=1}^r v_{ij} x_j$.
In other words, show that the algebraically independent subsets of these rational functions $f_i$ are indexed the same as the linearly independent subsets of the
$v_i$.
\end{enumerate}
}
\end{quote}

\bigskip
\item \emph{The Matroid Application Treasure Hunt.}
  Find as many applications of matroid theory as you can, both inside and outside of mathematics.
  One point for every application you find that has not been found by any other team; zero points for any duplicate application.
  I will collect the unique applications and make them available to all participants.

  \bigskip
\item
  In each of the classes or models of matroids discussed in class (linear / graphical / transversal / algebraic matroids; hyperplane / affine hyperplane arrangements, matroid polytopes),
  \begin{enumerate}
  \item
    describe independent sets, bases, circuits, cocircuits, and
    flats.
  \item
    For which of these entities can you rapidly see that they
    fulfill the corresponding axiom systems?  For which does it seem
    mysterious?
  \item
    Describe the dual matroid of a matroid in each of
    these situations.
  \end{enumerate}

  \bigskip
\item
  We have seen that a matroid can be given by its collections of independent sets, bases, circuits, cocircuits, flats, or its rank function.
  Find algorithms to convert between as many of these entities as you can.
  What is their combinatorial complexity?
 
  \bigskip

\item
  \emph{Independence complexes of matroids are vertex-decomposable.}
  Let $\Delta$ be a simplicial complex on the vertex set $E$.
  We do not assume that every $e \in E$ is actually used as a vertex of~$\Delta$.
  The concept of vertex-decomposability for a simplicial complex~$\Delta$ on the vertex set~$E$ is defined recursively:
  both the complex $\Delta = \emptyset$ having no faces at all (not even the empty face)
  and any complex $\Delta$ consisting of a single vertex are defined to be \defn{vertex-decomposable},
  and then $\Delta$~is said to be \defn{vertex-decomposable} if it is pure (all facets have the same dimension),
  and there exists a vertex $e \in E$ for which both its deletion and link
  \begin{align*}
    \del_\Delta (e) &:= \{F \in \Delta : e \notin F \}
    \\
    \link_\Delta (e) &:= \{F - \{e\} : e \in F \in \Delta\}
  \end{align*}
  are vertex-decomposable complexes.

  \begin{enumerate}
  \item
    Show that vertex-decomposable complexes $\Delta$ are shellable.

    \begin{quote}
    \footnotesize\color{green!30!black} 
    Hint: Obviously you want to use induction.
    Shell the facets in the deletion of~$e$ first,
    then those in the star of~$e$, which is the cone over the link with apex~$e$.
    Formally, for a face $F$ of~$\Delta$,
    \begin{align*}
      \thestar_\Delta(F) &= \{ G\in\Delta: F\cup G\in\Delta\},
      \\
      \link_\Delta(F) &= \{ G\in\Delta: F\cup G\in\Delta, F\cap G=\emptyset\}.
    \end{align*}
  \end{quote}

  \medskip
  \item
  Show that for a matroid $M$ with $\Delta = \mathcal I(M)$ and any non-loop, non-coloop
  element $e \in E$,
  \begin{align*}
    \del_\Delta (e) &= \mathcal I(M \setminus e),
    \\
    \link_\Delta (e) &= \mathcal I(M/e).
  \end{align*}
  Deduce that independent set complexes $\mathcal I(M)$ of matroids are vertex-decomposable.
\end{enumerate}
\bigskip

\item
\emph{Representability of the Fano and non-Fano matroids.}
Show that the Fano matroid is coordinatizable only in characteristic 2,
and the non-Fano matroid is coordinatizable only in characteristic distinct from 2.
\begin{enumerate}
\item
  First show that in any coordinatization $V = \{a, b, c, d, e, f, g\}$ of either
  the Fano or non-Fano matroids, with elements labelled as
  \begin{center}
    \begin{tikzpicture}
      \begin{scope}[very thick]
        \draw[draw=white, double=black, thick] (0,0) circle (1);
        \draw[thin] (1,0) -- (1,0.0001);
        \draw[thin] (-1,0) -- (-1,-0.0001);

        \draw[thin] ({-1/sqrt(2)},{-1/sqrt(2)}) -- ({-1/sqrt(2)+0.001},{-1/sqrt(2)-0.001});
        \draw[thin] ({1/sqrt(2)},{-1/sqrt(2)}) -- ({1/sqrt(2)+0.001},{-1/sqrt(2)+0.001});

        \node (0) at (0,0) {g};

        \foreach \i/\j/\k in {0/{b}/{d},1/{a}/{e},2/{c}/{f}}{%
          \node[fill=white] (\i;0) at ({2*cos(-30+120*\i)},{2*sin(-30+120*\i)}) {\(\j\)};
          \node[fill=white] (\i;1) at ({cos(30+120*\i)},{sin(30+120*\i)}) {\(\k\)};
        }
        \foreach \i/\j/\k/\l in {0/0/0/1,0/1/1/0,1/0/1/1,1/1/2/0,2/0/2/1,2/1/0/0}{%
          \draw[thin] (\i;\j) -- (\k;\l);
        }
        \draw[thin] (0) -- (0;1);
        \draw[thin] (0) -- (1;1);
        \draw[thin] (2;1) -- (0);
        \draw[thin] (0) -- (0;0);
        \draw[thin] (0) -- (1;0);
        \draw[thin] (0) -- (2;0);
      \end{scope}
    \end{tikzpicture}
    \qquad
    \begin{tikzpicture}
      \begin{scope}[very thick]
    %    \draw[draw=white, double=black, thick] (0,0) circle (1);
        \draw[thin] (1,0) -- (1,0.0001);
        \draw[thin] (-1,0) -- (-1,-0.0001);

        \draw[thin] ({-1/sqrt(2)},{-1/sqrt(2)}) -- ({-1/sqrt(2)+0.001},{-1/sqrt(2)-0.001});
        \draw[thin] ({1/sqrt(2)},{-1/sqrt(2)}) -- ({1/sqrt(2)+0.001},{-1/sqrt(2)+0.001});

        \node (0) at (0,0) {g};

        \foreach \i/\j/\k in {0/{b}/{d},1/{a}/{e},2/{c}/{f}}{%
          \node[fill=white] (\i;0) at ({2*cos(-30+120*\i)},{2*sin(-30+120*\i)}) {\(\j\)};
          \node[fill=white] (\i;1) at ({cos(30+120*\i)},{sin(30+120*\i)}) {\(\k\)};
        }
        \foreach \i/\j/\k/\l in {0/0/0/1,0/1/1/0,1/0/1/1,1/1/2/0,2/0/2/1,2/1/0/0}{%
          \draw[thin] (\i;\j) -- (\k;\l);
        }
        \draw[thin] (0) -- (0;1);
        \draw[thin] (0) -- (1;1);
        \draw[thin] (2;1) -- (0);
        \draw[thin] (0) -- (0;0);
        \draw[thin] (0) -- (1;0);
        \draw[thin] (0) -- (2;0);
      \end{scope}
    \end{tikzpicture},
  \end{center}
you
can use the action of $\GL_3 (\kk)$ along with scaling of individual vectors to assume
that the representing matrix has columns looking like this:
\[
  \kbordermatrix{
    & a & b & c & d & e & f & g\\
    & 1 & 0 & 0 & 1 & 1 & 0 & \gamma\\
    & 0 & 1 & 0 & 1 & 0 & \alpha & \delta\\
    & 0 & 0 & 1 & 0 & 1 & \beta & \epsilon
  }
\]
\item
  Use some of the matroid dependencies to show that $\gamma=\delta=\epsilon$, and hence by
  scaling, $\gamma = \delta = \epsilon = 1$.
\item  Use some more of the matroid dependencies to show that $\alpha = \beta$.
\item Use the last matroid dependence in the Fano matroid (and its absence in the non-Fano matroid)
  to decide whether or not the characteristic of $\kk$ is $2$.
\end{enumerate}

\bigskip

\item
  Let $L_{n,d}$ be the set of \emph{lattice points} (i.e., points all whose coordinates are integers) in the following $(d-1)$-dimensional simplex in $\RR^d$:
  \[
    \Delta_{d-1}(n)
    \ = \
    \big\{
    (x_1,x_2,\dots,x_d)\in\RR^d : x_1+x_2+\cdots + x_d = n-1
    \text{ and } x_i\ge0 \text{ for all } i
    \big\}.
  \]
  For $1\le k\in\NN$ a \emph{simplex of size~$k$} is a parallel translate of $L_{k,d}$.
  For any subset $I\subset L_{n,d}$, a simplex of size~$k$ is \emph{$I$-saturated} if it contains exactly $k$~points of~$I$.

  \medskip
  \begin{enumerate}[(a)]
  \item
    Let $S,S'$ be two $I$-saturated simplices with $S\cap S'\ne\emptyset$, and
    let $S\vee S'$ be the smallest simplex containing~$S$~and~$S'$.
    Then the simplices $S\cap S'$ and $S\vee S'$ are also $I$-saturated.
  \end{enumerate}

  \medskip
  Let $\cI_{n,d}$ be the collection of subsets $I$ of $L_{n,d}$ such that for every $k\in\NN$ with $1\le k\le n$,
  every parallel translate of $L_{n,k}$ contains at most $k$~points of~$I$, cf.~Figure~\ref{fig:holes}.

  \medskip
  \begin{enumerate}[(a)]
    \setcounter{enumii}{1}
  \item Show that $\cI_{n,d}$ is the collection of independent sets of a matroid.
    \begin{figure}[htbp]\centering
      
      \begin{tikzpicture}
        \foreach \x in {0,1,2,3}{
          \draw[draw=blue,thick] (\x,0) circle (.05);
        }
        \foreach \x in {0.5,1.5,2.5}{
          \draw[draw=blue,thick] (\x,.866) circle (.05);
        }
        \foreach \x in {1,2}{
          \draw[draw=blue,thick] (\x,1.732) circle (.05);
        }
        \draw[draw=blue,thick] (1.5,2.598) circle (.05);

        \foreach \x/\y in {2/0,.5/.866,2.5/.866,1.5/2.598}{
          \draw[draw=black,fill=blue,thick] (\x,\y) circle (.1);
        }
        \draw (1.5,3.031) circle(0);
        \draw (-.5,-.433) circle(0);
        \draw (3.5,-.433) circle(0);
      \end{tikzpicture}
      \qquad
      \begin{tikzpicture}
        \foreach \x in {0,1,2,3}{
          \draw[draw=blue,thick] (\x,0) circle (.05);
        }
        \foreach \x in {0.5,1.5,2.5}{
          \draw[draw=blue,thick] (\x,.866) circle (.05);
        }
        \foreach \x in {1,2}{
          \draw[draw=blue,thick] (\x,1.732) circle (.05);
        }
        \draw[draw=blue,thick] (1.5,2.598) circle (.05);

        \foreach \x/\y in {2/0,.5/.866,2.5/.866,1.5/2.598}{
          \draw[draw=black,fill=blue,thick] (\x,\y) circle (.1);
        }

        \draw[green!20!black] (-.5,-.433)--(3.5,-.433);
        \draw[green!20!black] (-.5,-.433)--(1.5,3.031);
        \draw[green!20!black] (3.5,-.433)--(1.5,3.031);

        \draw[green!20!black] (0,.433)--(3,.433);
        \draw[green!20!black] (.5,1.299)--(1.5,1.299);
        \draw[green!20!black] (1,2.165)--(2,2.165);
        
        \draw[green!20!black] (.5,-.433)--(1,.433);
        \draw[green!20!black] (1.5,1.299)--(2,2.165);
        \draw[green!20!black] (1.5,-.433)--(2.5,1.299);

        \draw[green!20!black] (.5,1.299)--(1,.431);
        \draw[green!20!black] (1.5,1.299)--(2,.431);
        \draw[green!20!black] (2,.431)--(2.5,-.433);
      \end{tikzpicture}      
      
      \caption{
        \emph{Left:} The lattice points $L_{4,3}$ in~$\Delta_2(4)$. The filled circles correspond to a basis (a maximal independent set) of $\cI_{4,3}$.
        \emph{Right:} A tesselation of $T(4)$ with $4$~triangles (``holes'') and the rest rhombi of angles $60^\circ$ and~$120^\circ$.
      } \label{fig:holes}
    \end{figure}
    \medskip
  \item Let $T(n)$ be an equilateral triangle with side length $n$.
    Suppose we want to tile $T(n)$ using unit rhombi with angles equal to $60^\circ$ and $120^\circ$.
    Show that this is impossible.
    \begin{quote}
      \footnotesize\color{green!30!black} 
      Hint: Cut $T(n)$ into $n^2$ unit equilateral triangles.
      How many of these point upward?
      How many downward?
    \end{quote}

  \medskip
\item Suppose that we make $n$~holes in the triangle $T(n)$, by cutting out $n$~of the upward triangles.
  Show that it may or may not be possible to tile the remaining shape with rhombi.

  \medskip
\item Show that the possible locations of $n$ holes for which a rhombus tiling of the ``holey'' triangle~$T(n)$ exists
  correspond to the bases of the matroid~$\cI_{n,3}$. 
\end{enumerate}

\bigskip
\item
  For each the affine Gale diagrams of Figure~\ref{fig:Gale0},
  determine the dimension and write down the vertex sets of the facets of their corresponding polytopes.
  Gray points are pyramid points.
  \begin{figure}[htbp]
    \centering
    \begin{tikzpicture}
      \draw[black,fill=black] (0,0) circle(.1);
      \draw[black,fill=black] (.18,.1) circle(.1);
      \draw[black,fill=white] (.18,-.1) circle(.1);
      \draw[black,fill=white] (0,-.18) circle(.1);
      \draw[black,fill=black] (-.18,.1) circle(.1);
      \draw[black,fill=black] (-.18,-.1) circle(.1);

      \draw[black,fill=black] (2,0) circle(.1);
      \draw[black,fill=white] (2.18,.1) circle(.1);
      \draw[black,fill=white] (2.18,-.1) circle(.1);
      \draw[black,fill=white] (2,-.18) circle(.1);
      \draw[black,fill=black] (2-.18,.1) circle(.1);
      \draw[black,fill=black] (2-.18,-.1) circle(.1);

      \draw[black,fill=black] (4,0) circle(.1);
      \draw[black,fill=black!50!white] (4.18,.1) circle(.1);
      \draw[black,fill=white] (4.18,-.1) circle(.1);
      \draw[black,fill=white] (4,-.18) circle(.1);
      \draw[black,fill=black] (4-.18,.1) circle(.1);
      \draw[black,fill=black] (4-.18,-.1) circle(.1);

      \draw[black,fill=black] (6,0) circle(.1);
      \draw[black,fill=black!50!white] (6.18,.1) circle(.1);
      \draw[black,fill=white] (6.18,-.1) circle(.1);
      \draw[black,fill=white] (6,-.18) circle(.1);
      \draw[black,fill=black] (6-.18,.1) circle(.1);
      \draw[black,fill=black!50!white] (6-.18,-.1) circle(.1);

    \end{tikzpicture}
    \caption{Four affine Gale diagrams}
    \label{fig:Gale0}
  \end{figure}

\bigskip
\item Consider the convex polytopes described by the following affine Gale diagrams in Figure~\ref{fig:Gale1}.
  \begin{figure}[htbp]
    \centering
    \begin{tikzpicture}
      \draw (0,1.5)--(1,1.75)--(2,1.75)--(3,1.5);
      \draw (0,0)--(1,-.25)--(2,-.25)--(3,0);
      \draw[black,fill=black] (0,1.5) circle(.1);
      \draw[black,fill=white] (1,1.75) circle(.1);
      \draw[black,fill=black] (2,1.75) circle(.1);
      \draw[black,fill=white] (3,1.5) circle(.1);
      \draw[black,fill=white] (0,0) circle(.1);
      \draw[black,fill=black] (1,-.25) circle(.1);
      \draw[black,fill=white] (2,-.25) circle(.1);
      \draw[black,fill=black] (3,0) circle(.1);
    \end{tikzpicture}
    \qquad\qquad
    \begin{tikzpicture}
      \draw (0,1.5)--(1,1.25)--(2,1.25)--(3,1.5);
      \draw (0,0)--(1,-.25)--(2,-.25)--(3,0);
      \draw[black,fill=black] (0,1.5) circle(.1);
      \draw[black,fill=white] (1,1.25) circle(.1);
      \draw[black,fill=black] (2,1.25) circle(.1);
      \draw[black,fill=white] (3,1.5) circle(.1);
      \draw[black,fill=white] (0,0) circle(.1);
      \draw[black,fill=black] (1,-.25) circle(.1);
      \draw[black,fill=white] (2,-.25) circle(.1);
      \draw[black,fill=black] (3,0) circle(.1);
      \draw[dashed] (2,1.25) -- (-.5,.625);
      \draw[dashed] (1,1.25) -- (3.5,.625);
    \end{tikzpicture}    
    \caption{Two affine Gale diagrams}
    \label{fig:Gale1}
  \end{figure}

  \begin{enumerate}
  \item
    Show that both convex polytopes have dimension~$4$ and $8$~vertices.
  \item
    Using the Gale diagram,
    show that both polytopes are \defn{$2$-neighborly},
    which means that all $\binom{8}{2}$~edges lie on the convex hull,
    or equivalently, that their graph is the complete graph~$K_8$.
  \item
    From the Gale diagram, deduce all $2$- and $3$-dimensional faces of these polytopes.
  \item
    Draw the \defn{dual graph} of these polytopes, which is the graph that has the $3$-dimensional faces (``facets'') as nodes,
    and in which two nodes are adjacent iff the corresponding $3$-dimensional faces intersect in a $2$-dimensional face.
  \item
    Show that the two polytopes are not \defn{combinatorially equivalent}, which means that there is no bijection between their vertex sets
    that induces a bijection between the sets of faces.
    \begin{quote}
      \footnotesize\color{green!30!black} 
      Hint: The left diagram is that of the \emph{cyclic} polytope $C_4(8)$. We met its smaller cousin, $C_4(7)$, in an in-class exercise.
    \end{quote}
    
\end{enumerate}

\bigskip
\item
  Consider the affine Gale diagram of Figure~\ref{fig:Gale2}.
  \begin{figure}[htbp]
    \centering
    \begin{tikzpicture}
      \draw (-2,3)--(0,0)--(2,3);
      \draw (-1,1.5)--(2,3);
      \draw (-2,3)--(1,1.5);
      \draw[black,fill=white] (-2,3) circle(.1);
      \draw[black,fill=black] (2,3) circle(.1);
      \draw[black,fill=black] (0,2) circle(.1);
      \draw[black,fill=black] (0,0) circle(.1);
      \draw[black,fill=black] (-1.05,1.55) circle(.1);
      \draw[black,fill=white] (-.99,1.48) circle(.1);
      \draw[black,fill=white] (1.05,1.6) circle(.1);
      \draw[black,fill=white] (.99,1.48) circle(.1);

      \node at (-2.3,3) {$2$};
      \node at (2.3,3) {$6$};
      \node at (-1.3,1.6) {$1$};
      \node at (-1.1,1.2) {$4$};
      \node at (1.3,1.6) {$7$};
      \node at (1.1,1.2) {$8$};
      \node at (.3,0) {$5$};
      \node at (0,2.3) {$3$};
    \end{tikzpicture}
    \caption{Another affine Gale diagram}
    \label{fig:Gale2}
  \end{figure}
  \begin{enumerate}
  \item
    Show that it represents a $4$-dimensional polytope with $8$~vertices.
  \item
    Show that the polytope has $9$~facets: four tetrahedra, four square pyramids, and an octahedron $235678$.
    Write down their vertex sets.
  \item
    Every Gale diagram with the same positive circuits has $7$ and $8$ on the same point.
  \item
    Therefore, in every combinatorially equivalent polytope the vertices $2356$ of the octahedron facet are coplanar.
    In consequence, the shape of the octahedron facet cannot be prescribed arbitrarily.
    
  \end{enumerate}
\end{enumerate}

\end{document}
