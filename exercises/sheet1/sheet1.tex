\documentclass[11pt]{amsart}

\usepackage{a4wide}
\usepackage{paralist}
\usepackage{url}
\usepackage{bbm}
\usepackage{fancyvrb}
\usepackage{xcolor}

\newcommand{\cA}{\mathcal{A}}
\newcommand{\cS}{\mathcal{S}}
\DeclareMathOperator{\conv}{conv}
\DeclareMathOperator{\New}{New}
\DeclareMathOperator{\area}{area}
\newcommand{\RR}{\mathbbm{R}}
\newcommand{\CC}{\mathbbm{C}}

\newcommand{\defn}[1]{{\color{blue}#1}}
\newcommand{\alert}[1]{\textbf{\color{red}#1}}

\DefineShortVerb{\;}
\renewcommand{\FancyVerbFormatCom}{\color{green!30!black}}

\begin{document}
\begin{center}
\textbf{\sffamily
   Discrete and Algorithmic Geometry }

\medskip
   Julian Pfeifle,
   UPC, 2019
\end{center}

\bigskip

\begin{center}
  \textbf{\sffamily Sheet 0}

\bigskip
\bigskip
 due on Tuesday, November 12, 2019

\end{center}

\bigskip

\section*{Reading}

\begin{enumerate}
  \setcounter{enumi}{-5}
\item
  Learn about the versioning software ;git;, and practice until you become comfortable using it.
  Then check out the repository of this course using
  \begin{center}
    ;git clone git@gitlab.mat-apl.upc.edu:julian.pfeifle/2019-dag-upc; .
  \end{center}

  \medskip
\item
  Create a new branch ;your-name-cv; in the repository, and edit the file ;participants.tex; to include a short cv and some information about your mathematical interests.
  Then ;commmit; and ;push; your changes, and create a pull request at ;gitlab.mat-apl.upc.edu; so that all the different stories may be merged.
  
  \medskip
\item
  Learn about public key cryptography and the use, advantages and disadvantages of the software ;gpg;.

  \medskip
\item
  Organize into teams of 2--3 people to work on the exercises, and
  edit  ;participants.tex; to reflect these changes.
  As always, ;commit; and ;push; your changes, and create a pull request at ;gitlab.mat-apl.upc.edu;.

  \medskip
\item
  Read up on two programming languages of your choice that you and your team will use in this course.
  One of these should be a scripting language for rapid iteration, the other a compiled language for efficiency.
  If you have never programmed before, a good choice for a scripted language is ;python;/;sage;, and a good choice for a compiled language is ;julia;.
  If you already know some languages, take the opportunity to learn a new one! Some suggestions are ;c++;, ;perl;/;raku;, ;rust;, ;haskell;.
\end{enumerate}

\newpage
\begin{center}
  \textbf{\sffamily Sheet 1}

 \bigskip
\bigskip
\bigskip
due on Tuesday, November 19, 2019
\end{center}

\bigskip

\section*{Writing}

\bigskip

\begin{quote}\small
  To submit your solutions to these exercises,
  \begin{itemize}[$\quad\triangleright$]
  \item create a new branch ;your-team-sheet-1;,
  \item create a subdirectory ;exercises/sheet1/your-awesome-team-name/;,
  \item and put your solutions to the exercises into a ;.pdf; file into that directory.
  \item Now encrypt this ;.pdf; using ;julian.pfeifle@upc.edu.public.gpg.key;, and
  \item ;add;, ;commit; and ;push; \alert{only this encrypted pdf, not the original} ;.tex;
  \item and create a pull request.
  \end{itemize}
  You will be graded collectively on these exercises, and individually in the final exam.

  \alert{Exercises not submitted via this mechanism will not be graded.}
\end{quote}

\bigskip

Let $([n],\mathcal I)$ be a matroid on the ground set $[n]=\{1,2,\dots,n\}$ and whose independent sets are~$\{I:I\in\mathcal I\}$.
Recall the following definitions:

\begin{itemize}[$\triangleright$]
\item
  For any proper subset $S\subset[n]$, the \defn{deletion}
  $M\setminus S$ is the matroid on the ground set $[n]\setminus S$ whose
  independent sets are $\{I\subset[n]\setminus S : I\in\mathcal I\}$.

\item
  The \defn{dual matroid} $M^\star$ of~$M$ is the matroid on $[n]$ where $I$~is a basis iff $[n]\setminus I$ is a basis~of~$M$.

\item
  If $S\subset[n]$, then the \defn{contraction} of $M$ with respect to $S$ is $M/S = (M^\star\setminus S)^\star$.
  
\end{itemize}

\bigskip
\begin{enumerate}
\item Why does this notion of contraction agree with the notion of contraction in graph theory?

  \bigskip
\item Prove that if a matroid $M$ is realizable over a ground field $\mathbbm k$, then the dual matroid~$M^\star$ is also realizable over~$\mathbbm k$.

  \medskip
  \begin{quote}\small
  \emph{Hint.} Suppose that $M$ has rank~$d$ and $n$~elements.
  After a change of basis, $M$~can be represented by the $d\times n$ matrix $A=[I|B]$, where $I$~is the $d\times d$ identity matrix, and $B$~has size $d\times(n-d)$.
  Now find a matrix that represents~$M^\star$.
\end{quote}

\bigskip
\item
  Consider the matroid given by the columns of
  \[
    \begin{bmatrix}
      0 & 1 & 1 & 0 & 1 \\
      1 & 0 & 1 & 0 & 1 \\
      0 & 1 & 1 & 1 & 0
    \end{bmatrix},
  \]
  and compute its dual, and the contraction along some subsets of your choice.
\end{enumerate}

\bigskip

\section*{Coding}



\end{document}
