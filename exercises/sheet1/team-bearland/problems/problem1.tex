\textbf{\textit{(1) True or false?}}

\vspace{3pt}

\hspace{5pt} \textbf{\textit{(a) This notion of contraction agrees with the notion of contraction in graph theory.}}

\vspace{3pt}


\begin{figure}[h!]
    \centering
    \begin{tikzpicture}[thick,scale=0.8]%
        \draw (0,0) node[style=vertice]{} -- ++ (0.5, 0.5) node[above] {e} -- (1,1) node[style=vertice]{};
    \end{tikzpicture}
\end{figure}

\hspace{5pt} \textbf{\textit{(b) $M_{G^\star} = \left(M_G\right)^\star$, if G is a planar graph and $G^\star$ its dual planar graph.}}

\vspace{10pt}

\textbf{\textit{(2) Prove that if a matroid M is realizable over a ground field $\mathbbm{k}$, then the dual matroid $M^\star$ is also realizble over $\mathbbm{k}$.}}

\vspace{10pt}

\textbf{\textit{(3) Consider the matroid M realized by the columns of the matrix}}
$$
A = \left[
    \begin{array}{ccccc}
        0 & 1 & 1 & 0 & 1 \\
        1 & 0 & 1 & 0 & 1 \\
        0 & 1 & 1 & 1 & 0
    \end{array}
\right]
$$
\textbf{\textit{ Compute a realization of $M^\star$, and some contractions of M of your choosing.}}

\vspace{5pt}

We will first compute a realization of matrix A.
To do so, we will swap columns two and four and perform a change of basis to basis $\mathcal{B} = \left\lbrace \left( \begin{array}{c} 0 \\ 1 \\ 0 \end{array} \right), \left( \begin{array}{c} 1 \\ 0 \\ 1 \end{array} \right), \left( \begin{array}{c} 1 \\ 1 \\ 1 \end{array} \right) \right\rbrace$. We then have,
\begin{equation*}
    \begin{split}
        A & = \left[
            \begin{array}{ccccc}
                0 & 1 & 1 & 0 & 1 \\
                1 & 0 & 1 & 0 & 1 \\
                0 & 1 & 1 & 1 & 0
            \end{array}
        \right]
        \longrightarrow
        \left[
            \begin{array}{ccccc}
                0 & 0 & 1 & 1 & 1 \\
                1 & 0 & 1 & 0 & 1 \\
                0 & 1 & 1 & 1 & 0
            \end{array}
        \right]
        \overset{\mathcal{B}}{\longrightarrow}
        \bar{A} = \left[
            \begin{array}{ccccc}
                1 & 0 & 0 & -1 & 0 \\
                0 & 1 & 0 & 0 & -1 \\
                0 & 0 & 1 & 1 & 1
            \end{array}
        \right]
        \hspace{10pt} \text{ which also represents $M$.} \\[10pt]
        \bar{A}B^T & = 0 \Longrightarrow 
        B^T = \left[
            \begin{array}{cc}
                -1 & 0  \\
                0  & -1 \\
                1  & 1 \\
                -1 & 0 \\
                0 & -1
            \end{array}
        \right] \hspace{5pt} \text{ (for example) } \Longrightarrow
        B = \left[
            \begin{array}{ccccc}
                -1 & 0 & 1 & -1 & 0 \\
                0 & -1 & 1 & 0 & -1
            \end{array} 
        \right] \hspace{5pt} \text{ which is $M^*$.}
    \end{split}
\end{equation*}

Now we will compute a contraction of $M$, $M / 1$, using a deletion in its dual matroid, $M^*$, $(M^* \backslash 1)^*$.
\begin{equation*}
    \begin{split}
        B \backslash 1 & = \left[
            \begin{array}{cccc}
                0 & 1 & -1 & 0 \\
                -1 & 1 & 0 & -1 
            \end{array} 
        \right] \text{ ;} \hspace{5pt}
        (B  \backslash 1)\cdot c^T = 0 \Rightarrow 
        c^T = \left[
                \begin{array}{cc}
                    0 & 1 \\
                    1 & 1 \\
                    1 & 1 \\
                    1 & 0
                \end{array} 
            \right] \Rightarrow
        c = \left[
                \begin{array}{cccc}
                    0 & 1 & 1 & 1 \\
                    1 & 1 & 1 & 0
                \end{array} 
            \right] \hspace{3pt} \text{which is $M / 1$.}
    \end{split}
\end{equation*}

An alternative method for computing $ M / 1$ is doing the projection in the corresponding direction. In particular, to find $ M / 1$ it suffices to project $A$ in direction $\left( \begin{array}{ccc} 0 & 1 & 0 \end{array} \right)^T$ and get rid of loops. In particular,
\begin{equation*}
    \begin{split}
        A & = \left[
            \begin{array}{ccccc}
                0 & 0 & 1 & 1 & 1 \\
                1 & 0 & 1 & 0 & 1 \\
                0 & 1 & 1 & 1 & 0
            \end{array}
        \right]
        \overset{\text{\footnotesize{Project}}}{\longrightarrow}
        \left[
            \begin{array}{ccccc}
                0 & 0 & 1 & 1 & 1 \\
                0 & 1 & 1 & 1 & 0
            \end{array}
        \right]
        \overset{\text{\footnotesize{No Loops}}}{\longrightarrow}
        \left[
            \begin{array}{ccccc}
                0 & 1 & 1 & 1 \\
                1 & 1 & 1 & 0
            \end{array}
        \right] \hspace{5pt} \text{ which again is $ M / 1$.}
    \end{split}
\end{equation*}
