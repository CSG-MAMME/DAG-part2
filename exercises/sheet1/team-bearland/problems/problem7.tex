\textbf{\textit{(7) In each of the classes or models of matroids discussed in class (linear / graphical / transversal / algebraic matroids; hyperplane / affine hyperplane arrangements, matroid polytopes),}}

\hspace{5pt}\textbf{\textit{(a) describe independent sets, bases, circuits, cocircuits, and flats.}}

\vspace{3pt}

\begin{tabularx}{\textwidth}{X|X|X|X|X|}
                 & Linear & Graphical & Transversal       & Algebraic   \\ \hline
Independent sets & Sets of linearly independent vectors & Acyclic subgraphs of a graph (i.e. forest subgraphs) & (Fixed a bipartite graph and one of its two stable sets) Vertices of the stable set which are incident to the edges of a matching of the graph & Sets of elements $\{\alpha_i\} \subset L$ that are algebraically independent and such that $K[\{\alpha_i\}] \subset L$ is algebraic \\ \hline
Bases            & Bases (in the linear algebra sense) & Spanning trees of a graph& Vertices of the stable set which are incident to a maximal matching & Minimal sets $\{\alpha_i\}$ that are algebraically independent and such that $K[\{\alpha_i\}] = L$ \\ \hline
Circuits         & Sets $\{v_1, \ldots, v_k\}$ such that $a_1 v_1 + \ldots + a_k v_k = 0$ for unique (up to multiplicative constant) $a_1, \ldots, a_k \neq 0$ (or, equivalently, such that no $k-1$ vectors out of the $k$ ones are on the same $(k-2)$-dimensional space) & Simple cycles & Sets of $k$ vertices of the stable set that have $k-1$ neighbours in total and such that any time you remove a vertex from the set there exists a perfect matching between the rest $k-1$ vertices and its $k-1$ neighbours. & \\ \hline
Cocircuits       & Sets $\{u_1, \ldots, u_k\}$ such that its orthogonal space has dimension $d-1$ ($d$ is the dimension of the total space), and any $k-1$ vectors from the set have an orthogonal space of dimension $d$ & Minimal cuts of the graph (minimal means that if you remove an edge of such cut it is no longer a cut) & Sets of $k$ vertices of the stable set such that every maximal matching of the bipartite graph is incident to some vertex in the set and for any $k-1$ vertices from the set there exists a maximal matching not incident to them & \\ \hline
\end{tabularx}
\begin{tabularx}{\textwidth}{X|X|X|X|X|}
& Linear & Graphical & Transversal       & Algebraic   \\ \hline
Flats            & Sets of vectors whose generated space is orthogonal to any other vector & Sets of edges whose maximum tree subgraph has $k$ edges, such that any time you add another edge then the maximum tree subgraph has $k+1$ edges & Sets of vertices from the stable set with a maximum matching between them and its neighbours of size $k$, such that any time you add another vertex there exists a matching of size $k+1$& \\ \hline                                     
\end{tabularx}
\vspace{3pt}

\hspace{5pt}\textbf{\textit{(b) For which of these entities can you rapidly see that they fulfill the corresponding axiom systems?  For which does it seem mysterious?.}}

\vspace{3pt}

For the case of independent sets and bases, it remains quite obvious for the cases of linear, graphical and transversal matroids.
For the case of circuits, it is only clear in the case of graphical matroids.

\vspace{3pt}

\hspace{5pt}\textbf{\textit{(c) Describe the dual matroid of a matroid in each of these situations.}}

\vspace{3pt}

\begin{tabularx}{\textwidth}{X|X}
Linear & Matroid in which the bases generate the subspace of $E$ orthogonal to the subspace generated by the bases of the primal \\ \hline
Graphical & Matroid in which the bases are the spanning trees of the dual graph (assuming the primal graph is planar) \\ \hline
Transversal & The independent sets of the dual give information about maximal independent sets of the bipartite graph`\\ \hline
Algebraic & 
\end{tabularx}