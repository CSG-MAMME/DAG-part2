\textbf{\textit{(7) In each of the classes or models of matroids discussed in class (linear / graphical / transversal / algebraic matroids; hyperplane / affine hyperplane arrangements, matroid polytopes),}}

\hspace{5pt}\textbf{\textit{(a) describe independent sets, bases, circuits, cocircuits, and flats.}}

\vspace{3pt}

\begin{tabularx}{\textwidth}{X|X|X|X|X}
\hline
                 & Linear & Graphical & Transversal       & Algebraic   \\ \hline
Independent sets & Sets of linearly independent vectors & Sets of edges that do not form any cycles & Matchings  & Sets of elements $\{\alpha_i\} \subset L$ that are algebraically independent and such that $K[\{\alpha_i\}] \subset L$ is algebraic \\ \hline
Bases            & Bases (in the linear algebra sense) & Spanning trees of a graph & Maximal matchings & Minimal sets $\{\alpha_i\}$ that are algebraically independent and such that $K[\{\alpha_i\}] = L$ \\ \hline
Circuits         & Sets $\{v_1, \ldots, v_k\}$ such that $a_1 v_1 + \ldots + a_k v_k = 0$ for unique (up to multiplicative constant) $a_1, \ldots, a_k \neq 0$ (or, equivalently, such that no $k-1$ vectors out of the $k$ ones are on the same $(k-2)$-dimensional space) & Simple cycles &   &  \\ \hline
Cocircuits       & Sets $\{u_1, \ldots, u_k\}$ of the orthogonal space such that no $k-1$ vectors out of the $k$ ones are on the same $(k-2)$-dimensional space & In case of planar graphs, minimal sequence of faces $f_0, f_1, \ldots f_k$ with $f_0 = f_k$ such that two consecutive faces are adjacent & & \\ \hline
Flats            & Sets of vectors that generate all the vector space & Connected graphs & Maximal matchings & Sets $\{\alpha_i\}$ that are algebraically independent and such that $K[\{\alpha_i\}] = L$ \\ \hline                                                 
\end{tabularx}
\vspace{3pt}

\hspace{5pt}\textbf{\textit{(b) For which of these entities can you rapidly see that they fulfill the corresponding axiom systems?  For which does it seem mysterious?.}}

\vspace{3pt}

For the case of independent sets and bases, it remains quite obvious for the cases of linear, graphical and transversal matroids.
For the case of circuits, it is only clear in the case of graphical matroids.

\vspace{3pt}

\hspace{5pt}\textbf{\textit{(c) Describe the dual matroid of a matroid in each of these situations.}}

\vspace{3pt}

\begin{tabularx}{\textwidth}{X|X}
Linear & Matroid in which the bases generate the subspace of $E$ orthogonal to the subspace generated by the bases of the primal \\ \hline
Graphical & Matroid in which the bases are the spanning trees of the dual graph (assuming the primal graph is planar) \\ \hline
Transversal & The independent sets of the dual give information about maximal independent sets of the bipartite graph`\\ \hline
Algebraic & 
\end{tabularx}