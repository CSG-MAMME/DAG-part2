\textbf{\textit{(3) The greedy algorithm always works for matroids.}}

\hspace{5pt}\textbf{\textit{(a) Show that Kruskal's greedy algorithm always finds a maximum weight independent set in a matroid $M = (E, \mathcal{I})$, regardless of the choice of weight function $\omega : E \mapsto \mathbb{R}_{>0}$. Recall that the greedy algorithm starts by setting $I \coloneqq \emptyset$, and next repeatedly chooses $y \in E \backslash I$ with $I \cup \lbrace y \rbrace \in \mathcal{I}$ and with $\omega(y)$ as large as possible. It stops if no such $y$ exists.}}

\vspace{3pt}

Let us rename the independent sets $I_0, I_1, \ldots, I_r$. Where $I_0 = \emptyset$, and $I_{j+1} = I_j \cup \{y_j\}$, for some $y_j \in E \setminus I_j$. Let us show $|I_j| = j$ by induction, and that, assuming $E$ is finite, there exists a final $I_r$, for some $r$ that satisfies $r = \max_{I \in \mathcal I} |I|$.

For $j = 0$, $|I_0| = |\emptyset| = 0$
For $j > 0$ assuming induction hypothesis $I_{j+1} = I_j \cup \{y_j\}$, for some $y_j \in E \setminus I_j$, and therefore $|I_{j+1}| = |I_j| + 1 = j + 1$. Assume that the last independendent set is $I_s$ for some $s$. Obviously, $s \leq r = \max_{I \in \mathcal I} |I|$, since $I_s \in \mathcal I$. If $s < r$, there will be a set $I'$ with $|I'| = r$. By the exchange axiom, there would be an $y_s \in I' \ I_s \subset E \ I_s$ such that $I_s \cup \{y_s\} = \mathcal I$, leading to a contradiction.

We are left with proving that the weight we obtain with the algorithm is indeed the maximum achievable. Let us show by induction that $I_j$ is the maximum weight we can achieve among all sets of $\mathcal I$ that have $j$ elements. %%%%%

\vspace{3pt}

\hspace{5pt}\textbf{\textit{(b) Show that that this property characterizes independent sets of matroids among all simplicial complexes. In other words, given a simplicial complex $\Sigma$ for which the greedy algorithm always works, regardless of the weight function~$\omega$, show that $\Sigma = \mathcal I(M)$ for some matroid~$M$.}}

\vspace{3pt}

WRITE HERE
