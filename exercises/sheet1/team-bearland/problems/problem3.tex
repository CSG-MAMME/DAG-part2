\textbf{\textit{(3) The greedy algorithm always works for matroids.}}

\hspace{5pt}\textbf{\textit{(a) Show that Kruskal's greedy algorithm always finds a maximum weight independent set in a matroid $M = (E, \mathcal{I})$, regardless of the choice of weight function $\omega : E \mapsto \mathbb{R}_{>0}$. Recall that the greedy algorithm starts by setting $I \coloneqq \emptyset$, and next repeatedly chooses $y \in E \backslash I$ with $I \cup \lbrace y \rbrace \in \mathcal{I}$ and with $\omega(y)$ as large as possible. It stops if no such $y$ exists.}}

\vspace{3pt}

WRITE HERE

\vspace{3pt}

\hspace{5pt}\textbf{\textit{(b) Show that that this property characterizes independent sets of matroids among all simplicial complexes. In other words, given a simplicial complex $\Sigma$ for which the greedy algorithm always works, regardless of the weight function~$\omega$, show that $\Sigma = \mathcal I(M)$ for some matroid~$M$.}}

\vspace{3pt}

WRITE HERE
