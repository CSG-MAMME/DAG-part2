\textbf{\textit{(3) The greedy algorithm always works for matroids.}}

\hspace{5pt}\textbf{\textit{(a) Show that Kruskal's greedy algorithm always finds a maximum weight independent set in a matroid $M = (E, \mathcal{I})$, regardless of the choice of weight function $\omega : E \mapsto \mathbb{R}_{>0}$. Recall that the greedy algorithm starts by setting $I \coloneqq \emptyset$, and next repeatedly chooses $y \in E \backslash I$ with $I \cup \lbrace y \rbrace \in \mathcal{I}$ and with $\omega(y)$ as large as possible. It stops if no such $y$ exists.}}

\vspace{3pt}

Let us rename the independent sets $I_0, I_1, \ldots, I_r$, where $I_0 = \emptyset$, and $I_{j+1} = I_j \cup \{y_j\}$, for some $y_j \in E \setminus I_j$ with $\omega y_j$ as large as possible. Let us show $|I_j| = j$ by induction, and that, assuming $E$ is finite, there exists a final $I_r$, for some $r$ that satisfies $r = \max_{I \in \mathcal I} |I|$.

For $j = 0$, $|I_0| = |\emptyset| = 0$
For $j > 0$, assuming induction hypothesis, $I_{j+1} = I_j \cup \{y_j\}$, for some $y_j \in E \setminus I_j$, and therefore $|I_{j+1}| = |I_j| + 1 = j + 1$. Assume that the last independendent set is $I_s$ for some $s$. Obviously, $s \leq r = \max_{I \in \mathcal I} |I|$, since $I_s \in \mathcal I$. If $s < r$, there will be a set $I'$ with $|I'| = r$. By the exchange axiom, there would be an $y_s \in I' \setminus I_s \subset E \setminus I_s$ such that $I_s \cup \{y_s\} \in \mathcal I$, leading to a contradiction since $|I_s \cup \{y_s\} \in \mathcal I| = s + 1 > s$.

We are left with proving that the weight we obtain with the algorithm is indeed the maximum achievable. Let us show by induction that $I_j$ is contained in an optimal basis $B$.

For the case $j = 0$, it is trivial, since $\emptyset = I_0 \subset B$. Now, applying induction hypothesis, if $I_j \subset B$, let us see what happens for $I_{j+1} = I_j \cup \{y_j\}$. If $y_j \in B$, $I_{j+1} \subset B$, so let us assume $y_j \notin B$. In this case $I_{j+1} \not\subset B$, but $I_{j+1} \subset B'$ for some basis $B'$ which is not optimal. By the exchange axiom, since $y_j \in B' \setminus B$, there is $y_j' \in B \setminus B'$ such that $B'' = B \setminus \{y_j'\} \cup \{y_j\}$ is a basis. $I_j \cup \{y_j'\}$ is independent as well since it is contained in a basis. By the construction of the algorithm, $\omega(y_j') \leq \omega(y_j)$. This implies that
\[
\sum_{b \in B} \omega(b) =
w(y_j') + \sum_{b \in B, b \neq y_j'} \omega(b) \leq
w(y_j) + \sum_{b \in B, b \neq y_j} \omega(b) =
\sum_{b \in B \setminus \{y_j'\} \cup \{y_j'\}} \omega(b) =
\sum_{b \in B''} \omega(b),
\]
Meaning that $B''$ is another optimal basis that contains $I_{j+1}$.

Finally, $I_r$, since its size will be the maximum set of an independent set, and therefore is a basis. Since a basis cannot be contained in another basis, $I_r$ will be the optimal basis.
\vspace{3pt}

\hspace{5pt}\textbf{\textit{(b) Show that that this property characterizes independent sets of matroids among all simplicial complexes. In other words, given a simplicial complex $\Sigma$ for which the greedy algorithm always works, regardless of the weight function~$\omega$, show that $\Sigma = \mathcal I(M)$ for some matroid~$M$.}}

\vspace{3pt}

Let $I_1, I_2 \in \mathcal I$ with $|I_2| = |I_1| + 1 = k + 1$, and let $\omega$ be defined as:
\[
\omega(e) = \left\{
\begin{array}{lr}
	\frac{k+1}{k+2}, & \text{for } e \in I_1 \\
	\frac{k}{k+1}, & \text{for } e \in I_2 \setminus I_1\\
    a(e), & \text{otherwise }
\end{array}
\right. ,
\]
Where the $a(e) > 0$ satisfy that $0 < \sum_{e \notin I_2} a(e) < k - \frac{k(k+1)}{k+2}$.

The algorithm will start building $I_1$ at the beginning since it will start with $\emptyset = I \subset I_1 \in \mathcal I$, and therefore, since $I \cup \{y\}$ with $y \in I_1$ will belong to $I_1$, by the second axiom of independent sets, $I \cup \{y\} \subset \mathcal I$, and therefore at the next step $I := I \cup \{y\} \subset I_1$. The same will happen while $I \subsetneq I_1$, each $y$ with maximum weight will be from $I_1$ and $I \cup \{y\}$ will be contained in $I_1 \in \mathcal I$ and therefore will also be contained in $\mathcal I$.

We want to see that the algorithm also chooses some $e \in I_2 \setminus I_1$. We can see that $\omega(I_2) = \sum_{x \in I_2} \omega(x) = (k+1) \frac{k}{k+1} = k$. If no element of $I_2 \setminus I_1$ was chosen, the algorithm would end up with an independent set of weight at most $\sum_{x \notin I_2 \setminus I_1} \omega(x) = \sum_{x \in I_1} \omega(x) + \sum_{x \notin I_1 \cup I_2} \omega(x) < \frac{k(k+1)}{k+2} + \sum_{e \notin I_1 \cup I_2} a(e) <  \frac{k(k+1)}{k+2} + k - \frac{k(k+1)}{k+2} = k$, contradicting that the algorithm finds a maximum weight independent set. Also, the first element after choosing all of $I_1$ will be of $I_2$ since the set of "available " elements is decreasing, and therefore, at the next step after having all of $I_2$, an element of maximum weight will have to be chosen, and therefore the chosen element will be part of $I_2 \ I_1$.

As we have seen before, when the algorithm ends the remaining set is a basis. Therefore it cannot stop at $I_1$, since a basis cannot be a subset of a different independent set, in this case $I_2$.
