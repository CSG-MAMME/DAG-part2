\textbf{\textit{(5) Prove (a) and (b) of the following implications for matroids, and illustrate them with a well-chosen example. Part (c) is optional and depends on your knowledge of algebra.}}

\begin{center}
\begin{tikzpicture}\small
  \tikzset{>=stealth}
    \node (algebraic) {Algebraic};
    \node [left=of algebraic] (linear) {Linear}
    edge[->] node[above] {(c)} (algebraic) ;
    \node [above left=of linear] (graphic) {Graphic}
     edge[->] node[auto] {(a)} (linear);
     \node [below left=of linear] (transversal) {Transversal}
     edge[->] node[below right] {(b)} (linear);
  \end{tikzpicture}
\end{center}

\vspace{3pt}

We will start the proof by recalling the definitions given in class for linear, transversal, and graphical matroids.
We will define them by means of their ground set $E$ and their independence sets $\I$.

\begin{definition}[Linear Matroid]
    Let $V$ be a vector space, then a linear matroid in $V$ is defined by
    \begin{equation*}
        \begin{split}
            E & = \lbrace \text{ finite subset of } V \rbrace \\
            \I & = \lbrace \text{ linearly independent subsets of } E \rbrace
        \end{split}
    \end{equation*}
\end{definition}

\begin{definition}[Graphical Matroid]
    Let $G = (V, E)$ be a graph,
    \begin{equation*}
        \begin{split}
            E & = \lbrace \text{ edges of } G \rbrace = E \\
            \I & = \lbrace \text{ edges of forests of } \mathcal{G} \rbrace
        \end{split}
    \end{equation*}
\end{definition}

\begin{definition}[Transversal Matroid]
    Let $G = (U, V, E)$ be a bipartite graph,
    \begin{equation*}
        \begin{split}
            E & = \lbrace \text{ bottom vertices of } G \rbrace = U \\
            \I & = \lbrace \text{ endpoints of maximal matchings } \mathcal{V} \rbrace
        \end{split}
    \end{equation*}
\end{definition}

To first prove (a), we want to build a linear matroid from a graphical one.
The first step is to choose our vector space, $U$.
Following the hint, let us define $U = \mathbb{R}^V$ whose standard basis is indexed by the vertices of $G = (V, E)$.
Then, if $V = \lbrace v_1, \dots, v_n \rbrace$ and $\lbrace e_{v_1}, \dots, e_{v_n}\rbrace$ is the standard basis in $U$, our ground and independence sets are defined as follows:
\begin{equation*}
    \begin{split}
        E = \lbrace e_{v_i} - e_{v_j} : v_iv_j \in E \rbrace ;
        \hspace{5pt} \I = \bigcup_{f \text{ forest}} \lbrace e_{v_i} - e_{v_j} : v_iv_j \in E(f) \rbrace
    \end{split}
\end{equation*}
It just remains to prove that $\I$ as defined are indeed linearly independent subsets of $E$.
Let's assume not, \textit{i.e.} one of such subset contains a linear dependence.
This is, for a forest $f$ in $G$ we have:
\begin{equation*}
    e_{v_k} - e_{v_l} = (e_{v_l} - e_{v_j}) + \dots + (e_{v_m} - e{v_n})
\end{equation*}
in this case it is clear that $k = n$ and there are two different paths to go from $v_l$ to $v_k$ what implies the existance of a cycle and hence $f$ is not a forest.
This contradicts our initial hypothesis and as a consequence $\I$ as defined is indeed an independence set what proves (a).

To prove (b) we will proceed in a similar manner.
In this case we have a bipartite graph $G = (E, F, H)$ where $E$ is the \textit{bottom} partition of vertices.
For our linear matroid we choose our vector space $V$ to be $\mathbbm{k}(X)^F$ with $\mathbbm{k}(X)$ defined as in the hint, and with standard basis indexed by vertices $f \in F$.
Then our ground, and independence sets are defined as follows:
\begin{equation*}
    \begin{split}
        E = \left\lbrace \left\lbrace \sum_{f \in F, ef \in H} x_{e,f} u_f \right\rbrace : e \in E \right\rbrace ;
        \hspace{5pt} \I = \bigcup_{\text{$m$ max. matching}} \lbrace e : ef \in m, e \in E, F \in F \rbrace
    \end{split}
\end{equation*}
It just remains to prove that the sets in $\I$ are indeed independent.
In the same way we did for (a), we will assume some set not to be independent, hence yielding the following depndence relation:
\begin{equation*}
    \sum_{f \in F, e_1f \in H} x_{e_1,f} u_f = \sum_{f \in F, e_2f \in H} x_{e_2,f} u_f  + \dots + \sum_{f \in F, e_kf \in H} x_{e_1,f} u_f 
\end{equation*}
given that, by construction, the field extension $\mathbbm{k}(X)$ is defined by trascendentals $\lbrace x_{e,f}$, the equality can only hold iff some trascendental (\textit{e.g.} $x_{e_1,f}$ appears more than once, hence not being $m$ originally a matching in $G$.
This contradiction proves in turn (b).
