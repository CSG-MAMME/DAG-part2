\documentclass[11pt]{amsart}

\usepackage{mathptmx}
\usepackage{a4wide}
\usepackage{paralist}
\usepackage{url}
\usepackage{hyperref}
\usepackage[utf8]{inputenc}  

\newcommand{\cA}{\mathcal{A}}
\newcommand{\cS}{\mathcal{S}}

\begin{document}
\begin{center}
\textbf{\sffamily
   Discrete and Algorithmic Geometry }

\medskip
   Julian Pfeifle,
   UPC, 2019
\end{center}

\bigskip

\begin{center}
  \textbf{\sffamily Participants}
\end{center}

\medskip

\section*{Overview}

\begin{center}
  \begin{tabular}[c]{lll}
    Your name
    & Your gitlab username
    & Your email
    \\\hline
    Carlos Segarra
    & \href{https://gitlab.com/csegarragonz}{@csegarragonz}
    & \texttt{\href{mailto:carlossegarragonzalez@gmail.com}{carlossegarragonzalez@gmail.com}}
    \\\hline
    Arnau Mir Fuentes
    & @UchihaArnau
    & \texttt{tux33@hotmail.com}
    \\\hline
    Adrian Tobar Nicolau
    & @atobarnicolau
    & \texttt{atobarnicolau@gmail.com}
    \\\hline
    Manuel Sánchez Torrón
    & @ManuTorron5
    & \texttt{manuelsancheztorron@gmail.com}
    \\\hline
    Júlia Folguera Profitós
    & @Folgs
    & \texttt{juliafolguera@gmail.com}
    \\\hline
    Jordi Pla Mauri
    & @jplam
    & \texttt{j.pla.mauri@gmail.com}
    \\\hline
    Víctor Martín Chabrera
    & @BaqablH
    & \texttt{victormartin96@outlook.com}
    \\\hline
    Biel Tura
    & \href{https://gitlab.com/bieltv3}{@bieltv3}
    & \texttt{\href{mailto:bieltv.3@gmail.com}{bieltv.3@gmail.com}}
    \\\hline
    Sebastià Mijares i Verdú
    & @sebastia.mijares
    & \texttt{sebastia.mijares@estudiant.upc.edu}
    \\\hline
    Eloi Torrents Juste
    & Eloitor
    & \texttt{eloi.torrentsj@gmail.com}
    \\\hline
    Daniel Gómez Barroso
    & @Ragger
    & \texttt{dgombarroso@gmail.com}
    \\\hline
    Sarah Zampa
    & sarah.zampa
    & zampa-sarah@hotmail.fr
    \\\hline
  \end{tabular}
\end{center}

\section*{Carlos Segarra}

%A short cv here. For example, you could write about the following:
My name is Carlos Segarra and I am a student enrolled in \textit{Graph Theory}, \textit{Discrete and Algorithmic Geometry}, and \textit{Concurrency, Parallelism, and Distributed Systems} from the Master in Research in Informatics (\href{https://www.fib.upc.edu/en/studies/masters/master-innovation-and-research-informatics}{MIRI-UPC}).

Formally, I hold a BSc in Mathematics and one in Electrical Engineering by the \textit{Centre de Formaci\'o Interdisciplin\`aria Superior} (\href{https://cfis.upc.edu}{CFIS-UPC}), and I am looking forward to strengthen my knowledge in discrete mathematics and cryptography.
For any further reference, CV, or other social media handles you can visit my \href{https://carlossegarra.com}{website}.

Moreover, I really enjoy programming, and this is in fact what I spent most of my days doing (also as a day-job).
In particular, I am well-versed in \textsc{C}, and \textsc{Python}.
But I have also worked with JVM-based languages like \textsc{Java} and \textsc{Scala}, other scripting languages like \textsc{bash}, and other more esoteric programming languages like \textsc{Erlang}, and \textsc{Haskell}. 
On a side note, I really like \TeX and \texttt{vim}, so I can give a hand with that if needed.
To check out a subset of the projects I am working on you can visit my \href{https://github.com/csegarragonz}{Github}.

In this course, I expect to gain a greater insight into how geometry problems are solved and applied to real world scenarios.
However, I expect this focus on applications not to come at the expense of mathematical rigor and correctness.
%without losing perspective on the correctness and theory begind the concepts we deploy.
It is this link between computer science and mathematics that I look forward to exploiting in the coming years as a PhD in the area of distributed systems, networking, and security.

\medskip


\section*{Arnau Mir}
My name is Arnau and I am enrolled in the courses of \textit{Graph Theory}, \textit{Discrete and Algorithmic Geometry}, \textit{Number Theory} and \textit{Commutative Algebra}

I obtained a degree in Mathematics by the University of Balearic Islands. On the one hand, my background in geometry was obtained coursing Euclidean Geometry, Affine Geometry, Differential Geometry and Topological Geometry.
On the other hand, I have a little background programming with languages like Java and Python, but I only spent one year improving my skills in both languages. 
I hope that in this subject I can improve my techniques to solve problems related to geometry, I have a special interest in the theoretical vision of these problems.

In relation to my future, I'm not sure which branch of mathematics I want to work on but  I am very motivated to learn new techniques to attack advanced math problems and I think the best option would be to pursue a PHD.

\section*{Adrián Tobar}


I obtained my degree in Mathematics from the University of the Balearic Islands (UIB) where the emphasis of my studies was on geometry, analysis and algebra.
Most of all, I enjoyed discrete mathematics and abstract algebra, and I disliked most of all probability and statistics.
I also did some subjects related to programming. I was introduced to C++ and Python. I also studied formal analysis of algorithms ( complexity, correctness,...)
I am interested in increasing my knowledge in algorithmic mathematics and related topics.

The programming related projects I have done were focused on making small projects by hand: visualize data, solve matrix problems, data mining (R, tidyverse),
primality tests,... Where the biggest project could be part of my degree thesis where I made and proved the correctness of some algorithms related to fuzzy logics.
I am interested in making bigger projects and solving real-life problems.

I want to learn more about algorithmic theory and be more prepared to be able to fight harder algorithmic problems in the future.  I would like to pursue a PhD and do research.



\medskip

\section*{Manuel Sánchez}

My name is Manuel Sánchez Torrón, and I did a dual degree in Mathematics and Computer Science in \textit{Universidad Complutense de Madrid}.
The subjects I enjoyed most were about discrete mathematics, differential and algebraic geometry and differential equations.

As a graduate in Computer Science, I have done several programming projects, in various languages such as C++, Java, Python, Matlab, Haskell, Prolog...
Among these projects I would like to highlight the one that I developed for my final thesis, which consisted in a program that numerically solves
systems of autonomous differential equations in the plane and in the space and plots the solutions, with a GUI implemented in Matlab 
(If you are interested, you can find it \href{https://github.com/ManuTorron5/TFG-app}{here})

About the future, I am interested in learning a little bit more about Mathematics that may be useful in Cryptography, Data Science or Artificial Intelligence,
and find a job related to one of these topics.


\section*{Júlia Folguera}

My name is Júlia Folguera and I hold a Bsc degree in Mathematics in \textit{Universitat de Barcelona}. I really enjoyed most of the subjects I took there,
but I'm more interested in numerical methods and applied mathematics in general. In my degree, I had to learn C and C++, but I also worked with Python, C\# and VisualBasic in an internship.

In my Bachelor, I used these programming languages to solve mathematical problems, usually continuous and not discrete problems, whereas in my internship
I used them mainly to work and process data.

I took this course because I enjoy both programming and geometry and expect it to be about a combination of mathematical theory and applications of that theory.
After finishing the Master, I would like to find a job in which I could apply some of the things that I'm currently learning.



\medskip

\section*{Jordi Pla Mauri}

My name is Jordi Pla and I hold a BSc in Biomedical Engineering from
\textit{Universitat Pompeu Fabra}.
Most of the mathematics I have done are Numerics and
mathematics applied to biology.
During my degree I had to code in C, MATLAB, Python, Javascript and Haskell.

Most of the programming projects I have done are related to implementing
Cellular Automata, Agent-based models, Finite Element solvers
and Image Processing pipelines.

I enrolled this course because I enjoy programming
and I would like to have a broader background in geometry.
After finishing the Master I will apply for a PhD in Biomedicine.

\section*{Víctor Martín Chabrera}

I have studied and obtained the degree in Mathematics from the \textit{Universitat Politècnica de Catalunya} (UPC). However, I took my first year studying Mathematics and Physics at the \textit{Universitat de Barcelona} (UB), and part of my second year studying the double degree program in Mathematics and Physics Engineering of the \textit{Centre de Formació Interdisciplinària Superior} (CFIS-UPC). I have also spent an important part of my undergraduate years training for Mathematics and Programming competitions, which has helped me learn skills for problem solving that are not normally taught in a classroom. I have also been very interested in the Collatz Conjecture for years, so I have made my degree thesis on the problem and found some interesting results, which I am trying to publish hopefully soon. 

The only remarkable project I have done so far is the one I am doing in my current job, consisting on developing a multi-modal routing algorithm, namely, an algorithm that allows a user to move from point A to point B in a city by combining all available mobility services (public transport, taxis, shared scooters, shared bikes...). I have many projects in mind for the future, but for now I am more focused on improving my skills in C++, which is the main language I know, and also get a little better in some other languages and tools like Python, Bash and Git.

From this course I expect, on the one hand, to gain the ability to solve problems related with geometry that I encounter in my work, and on the other, to improve my skills in geometry, which has always been one of my biggest weaknesses, both in the Mathematics and in the Programming side.

What I would want to be in some years is working where I am now (with the difference that for then I hope we have hundreds of thousands of users). There are obviously plans B and C...

You can see more \href{https://www.linkedin.com/in/victormartinchabrera}{here}.





\medskip

\section*{Biel Tura}
My name is Biel. I am a Telecommunication/Electrical engineer from the Universitat Politècnica de Catalunya (UPC - ETSETB). My major is in Audiovisual Systems: image and audio signal processing and Computer Vision / Artificial Intelligence. My exposure to the latter is my reason for me to join this master, as my goal is to improve my knowledge about Discrete Mathematics and Algorithms, which are used in many Machine Learning / Deep Learning applications. My background in mathematics comes from the basics mathematics courses of my bachelor and from a few specific Telecommunication applied math courses.

Apart from that, I have programmed in several languages, the main ones being \textsc{Python}, \textsc{C} and \textsc{JAVA}. Recently though, I have been experimenting with \textsc{CUDA} to generate Computer Vision applications with accelerated graphics in real-time.

This course has already introduced me to a few useful algorithms in terms of efficient computation. I think it is important to know and understand how to efficiently treat (search/order/manipulate) a set for my main field of interest (real-time AI) and I am looking forward to keep improving in this field.

Next semester I will be taking my second part of this master at the EPFL (Lausanne) and I will try to find an internship for the next academic year in Europe to, if possible, develop my master thesis.


\medskip

\section*{Sebastià Mijares i Verdú}
I studied Mathematics at the Universidad Complutense de Madrid, and my strengths (if they can be called that way) are mostly in Algebra, Set Theory, and Geometry, although I'm also comfortable in Analysis and Topology.
I'm not a particularly skillful programmer; in my degree we worked just a little bit with \textsc{Python} and \textsc{Matlab} (if that last one counts). It's a necessary skill for the age we live in, and I hope to improve
at it quickly.

Professionally, I've worked in administration for several years, in political affairs, and most recently in railway programmes, although in none of those cases I've worked in anything related to mathematics or programming.

In terms of mathematical research, my experience is (naturally) limited to the final parts of my degree and this master programme, with my interests so far focusing mostly on algebraic approaches to other problems and contexts,
such as graphs. I'd like to follow the path of research and teaching, although I may not be a strong enough student for that, so I hope to obtain the necessary skills for a professional career.
\medskip

\section*{Eloi Torrents Juste}

I studied maths at Autonomous University of Barcelona (UAB), where I took courses on analysis, number theory, differential equations and dynamical systems, among others.

I really enjoy number theory in general, so I would like to learn all the maths involved in the study of this field (mainly analysis, algebra combinatiorics and geometry).

I did some simple but quite strong AI's for different games (2048 and 4 in a row). I've tried to build an interpreter following the steps of the great tutorial \href{https://ruslanspivak.com/lsbasi-part1/}{Let’s build a simple interpreter} (which I would like to turn into a compiler some day). And I have worked on expanding a library to solve TSP-like problems.

I expect to learn more about algorithms and optimizing, but in the upcoming years I expect to focus on number theory.

\medskip

\section*{Daniel Gómez}

I am Daniel Gómez Barroso and I did a degree in Mathematics and Computer Science in \textit{Universidad Politécnica de Madrid}.

I am really interested in studying more in Computational Geometry, as I did my Degree Thesis focusing on this area and I want to do my Master Thesis also about it. I have a Collaboration Grant with the CGA group and I would like to mix that with my Master Thesis.

As a Computer Science student I have been working in some programming projects as a cinema manager, a production chain manager, a JavaScript processor, a Delaunay Triangulation and Voronoi Diagram implementation, and my Degree Thesis, a program meant to detect riverflows in a terrain using the Delaunay Triangulation of it. I have done them in Java or Python, but also have a little experience in some other languages like C, Haskell, Prolog, Matlab or Maple...

I am willing to follow this course as I think I can find some new things I am interested in (I have not studied anything about polytopes before) and some things that could even inspire me to continue studying in the future.

I honestly do not have any idea of where I see myself in 3-5 years, one possibility is to continue with a PhD in this area, but I have not decided yet.

\medskip


\section*{Lucia Costantini}

My name is Lucia Costantini and I am originally from Italy, though I finished high school in the United States and did my Bachelor's degree in Mathematics at \textit{Cardiff University}.
During my undergraduate studies, I mainly took subjects of pure mathematics, but I am looking forward to experiment a little with more applied topics. 
In the past year, I have written a short article on the role of Game Theory and specifically the Iterated Prisoner's Dilemma in animal cooperation and, later on, a project report on the topic of Random Number Generation, where I learned the bases of statisitcal programming language R. 
I really enjoyed making both of those and hope to have more essays or reports to do in the future.
I am still uncertain about what areas of mathematics interest me the most and what direction I will take, but I am considering applying for a PHD for the next year.

\medskip 

\section*{Sarah Zampa}

I studied and got my degree in France at the University of Lille. I studied mainly "pure" mathematics, and particularly enjoyed \textit{Topology} and \textit{Geometry}, but also studied Algebra, Analysis, Differential Calculus, Integration, etc. I also did a bit of programming (mainly using Python), with theory and practice.
I would like to intensify my knowledge in Topology, I am currently studying Differential Topology, but I am interested in a lot of subjects related to mathematics in general.
In the second year of my degree, I had a project based on image recognition to try and solve a newspaper's game (see "TrucMuche" from the newspaper "Voix du Nord" if interested). I also did a project on Conway's Game of Life which was very fun!
In n years' time, I would like to still be in the academic world, I definitely want to do a PhD, though I don't know if I will do it straight after this master, or if I will first do another master's year somewhere else to sharpen my knowledge.

\medskip 

\section*{next name}


\begin{itemize}
\item What kind of math have you studied?
\item What kind of math would you like to learn?
\item What programming projects have you done or would like to do?
\item What do you expect from this course?
\item Where do you want to be in three years' time? In five?
\end{itemize}


\medskip

\newpage
\section*{Teams for exercises}

\begin{center}
  \textbf{\sffamily Team roster}
\end{center}

\bigskip
\begin{center}
  \begin{tabular}[c]{r|l|l|l}
    Team name
    & Members
    & Programming language(s) we know
    & Ones we'll try to learn
    \\\hline
    \emph{Team Bearland}
    & Victor Mart\'in
      & \textsc{C++}, \textsc{C}, \textsc{Python} & \textsc{Rust}, \textsc{PureScript} \\
      & Pablo Oviedo & \\
      & Carlos Segarra &               
    \\\hline
    \emph{Terrible Island}
    & Adrian Tobar
    & C++, Python, & julia \\
    & Arnau Mir & \\
    & Sebastià Mijares i Verdú &
    \\\hline

                  
    

    \emph{Anna i Júlia}
    & Júlia Folguera
    & C, C++  & sage  \\
    & Anna Sopena &
    \\\hline
    \emph{Lost Engineers}
    & Biel Tura
    & \textsc{C}, \textsc{Python}  & julia  \\
    & Jordi Ventura &
    \\\hline



    \emph{Convex Group}
    & Mariola Barceló
    & \textsc{C/C++}, \textsc{Java}, \textsc{Python},\textsc{Matlab}  &  \textsc{Smalltalk}, \textsc{Scala}   \\
    & Edu Gonzalvo & \\
    & Eloi Torrents &  
    \\\hline
    
    \emph{Marta i Oriol}
    & Oriol Almirall
    & \textsc{C}, Sage & Python\\
    & Marta Herrerias &
    \\\hline

    \emph{A Team Has No Name}
    & Daniel Gomez 
    &  \textsc{Python},  \textsc{Haskell}  &  \textsc{julia} \\
    & Manuel Sanchez \\  
	\\\hline
	

	\emph{Dual Spaced Out}
    & Sarah Zampa 
    &  \textsc{Python} \\
    &  \textsc{C++} \\
    & Lucia Costantini \\  
	\\\hline
	

  \end{tabular}
\end{center}



\end{document}






