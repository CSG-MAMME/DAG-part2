\documentclass[11pt]{amsart}

\usepackage{mathptmx}
\usepackage{a4wide}
\usepackage{paralist}
\usepackage{url}
\usepackage{hyperref}
\usepackage[utf8]{inputenc}  

\newcommand{\cA}{\mathcal{A}}
\newcommand{\cS}{\mathcal{S}}

\begin{document}
\begin{center}
\textbf{\sffamily
   Discrete and Algorithmic Geometry }

\medskip
   Julian Pfeifle,
   UPC, 2019
\end{center}

\bigskip

\begin{center}
  \textbf{\sffamily Participants}
\end{center}

\medskip

\section*{Overview}

\begin{center}
  \begin{tabular}[c]{lll}
    Your name
    & Your gitlab username
    & Your email
    \\\hline
    Carlos Segarra
    & \href{https://gitlab.com/csegarragonz}{csegarragonz}
    & \texttt{\href{mailto:carlossegarragonzalez@gmail.com}{carlossegarragonzalez@gmail.com}}
    \\\hline
    Arnau Mir Fuentes
    & @UchihaArnau
    & \texttt{tux33@hotmail.com}
    \\\hline
    another name
    & another username
    & \texttt{another email}
=======
    Adrian Tobar Nicolau
    & @atobarnicolau
    & \texttt{atobarnicolau@gmail.com}
    \\\hline
    Manuel Sánchez Torrón
    & @ManuTorron5
    & \texttt{manuelsancheztorron@gmail.com}
    \\\hline
    Júlia Folguera Profitós
    & @Folgs
    & \texttt{juliafolguera@gmail.com}
>>>>>>> participants.tex
  \end{tabular}
\end{center}

\section*{Carlos Segarra}

%A short cv here. For example, you could write about the following:
My name is Carlos Segarra and I am a student enrolled in \textit{Graph Theory}, \textit{Discrete and Algorithmic Geometry}, and \textit{Concurrency, Parallelism, and Distributed Systems} from the Master in Research in Informatics (\href{https://www.fib.upc.edu/en/studies/masters/master-innovation-and-research-informatics}{MIRI-UPC}).

Formally, I hold a BSc in Mathematics and one in Electrical Engineering by the \textit{Centre de Formaci\'o Interdisciplin\`aria Superior} (\href{https://cfis.upc.edu}{CFIS-UPC}), and I am looking forward to strengthen my knowledge in discrete mathematics and cryptography.
For any further reference, CV, or other social media handles you can visit my \href{https://carlossegarra.com}{website}.

Moreover, I really enjoy programming, and this is in fact what I spent most of my days doing (also as a day-job).
In particular, I am well-versed in \textsc{C}, and \textsc{Python}.
But I have also worked with JVM-based languages like \textsc{Java} and \textsc{Scala}, other scripting languages like \textsc{bash}, and other more esoteric programming languages like \textsc{Erlang}, and \textsc{Haskell}. 
On a side note, I really like \TeX and \texttt{vim}, so I can give a hand with that if needed.
To check out a subset of the projects I am working on you can visit my \href{https://github.com/csegarragonz}{Github}.

In this course, I expect to gain a greater insight into how geometry problems are solved and applied to real world scenarios.
However, I expect this focus on applications not to come at the expense of mathematical rigor and correctness.
%without losing perspective on the correctness and theory begind the concepts we deploy.
It is this link between computer science and mathematics that I look forward to exploiting in the coming years as a PhD in the area of distributed systems, networking, and security.

\medskip

<<<<<<< participants.tex
\section*{Arnau Mir}
My name is Arnau and I am enrolled in the courses of \textit{Graph Theory}, \textit{Discrete and Algorithmic Geometry}, \textit{Number Theory} and \textit{Commutative Algebra}

I obtained a degree in Mathematics by the University of Balearic Islands. On the one hand, my background in geometry was obtained coursing Euclidean Geometry, Affine Geometry, Differential Geometry and Topological Geometry.
On the other hand, I have a little background programming with languages like Java and Python, but I only spent one year improving my skills in both languages. 
I hope that in this subject I can improve my techniques to solve problems related to geometry, I have a special interest in the theoretical vision of these problems.

In relation to my future, I'm not sure which branch of mathematics I want to work on but  I am very motivated to learn new techniques to attack advanced math problems and I think the best option would be to take a PHD.
=======
\section*{Adrián Tobar}


I obtained my degree in Mathematics from the University of the Balearic Islands (UIB) where the emphasis of my studies was on geometry, analysis and algebra.
Most of all, I enjoyed discrete mathematics and abstract algebra, and I disliked most of all probability and statistics.
I also did some subjects related to programming. I was introduced to C++ and Python. I also studied formal analysis of algorithms ( complexity, correctness,...)
I am interested in increasing my knowledge in algorithmic mathematics and related topics.

The programming related projects I have done were focused on making small projects by hand: visualize data, solve matrix problems, data mining (R, tidyverse),
primality tests,... Where the biggest project could be part of my degree thesis where I made and proved the correctness of some algorithms related to fuzzy logics.
I am interested in making bigger projects and solving real-life problems.

I want to learn more about algorithmic theory and be more prepared to be able to fight harder algorithmic problems in the future.  I would like to pursue a PhD and do research.
>>>>>>> participants.tex


\medskip

<<<<<<< participants.tex
\section*{Next name}
=======
\section*{Manuel Sánchez}

My name is Manuel Sánchez Torrón, and I did a dual degree in Mathematics and Computer Science in \textit{Universidad Complutense de Madrid}.
The subjects I enjoyed most were about discrete mathematics, differential and algebraic geometry and differential equations.

As a graduate in Computer Science, I have done several programming projects, in various languages such as C++, Java, Python, Matlab, Haskell, Prolog...
Among these projects I would like to highlight the one that I developed for my final thesis, which consisted in a program that numerically solves
systems of autonomous differential equations in the plane and in the space and plots the solutions, with a GUI implemented in Matlab 
(If you are interested, you can find it \href{https://github.com/ManuTorron5/TFG-app}{here})

About the future, I am interested in learning a little bit more about Mathematics that may be useful in Cryptography, Data Science or Artificial Intelligence,
and find a job related to one of these topics.


\section*{Júlia Folguera}

My name is Júlia Folguera and I hold a Bsc degree in Mathematics in \textit{Universitat de Barcelona}. I really enjoyed most of the subjects I took there,
but I'm more interested in numerical methods and applied mathematics in general. In my degree, I had to learn C and C++, but I also worked with Python, C\# and VisualBasic in an internship.

In my Bachelor, I used these programming languages to solve mathematical problems, usually continuous and not discrete problems, whereas in my internship
I used them mainly to work and process data.

I took this course because I enjoy both programming and geometry and expect it to be about a combination of mathematical theory and applications of that theory.
After finishing the Master, I would like to find a job in which I could apply some of the things that I'm currently learning.
>>>>>>> participants.tex


\begin{itemize}
\item What kind of math have you studied?
\item What kind of math would you like to learn?
\item What programming projects have you done or would like to do?
\item What do you expect from this course?
\item Where do you want to be in three years' time? In five?
\end{itemize}


\medskip

\section*{Next name}


\newpage
\section*{Teams for exercises}

\begin{center}
  \textbf{\sffamily Team roster}
\end{center}

\bigskip
\begin{center}
  \begin{tabular}[c]{r|l|l|l}
    Team name
    & Members
    & Programming language(s) we know
    & Ones we'll try to learn
    \\\hline
    \emph{Charlie's Angels}
    & Victor Mart\'in
      & \textsc{C++}, \textsc{C}, \textsc{Python} & \textsc{Rust}, \textsc{PureScript} \\
      & Pablo Oviedo & \\
      & Carlos Segarra &               
    \\\hline
    \emph{Terrible Island}
    & Adrian Tobar
    & C++, Python, & julia \\
    & Arnau Mir &
    \\\hline
<<<<<<< participants.tex
    \emph{next team name}
    & Member
    & language  & language, language  \\
    & Member \\
    & (Member)                
    
=======
    \emph{Anna i Júlia}
    & Júlia Folguera
    & C, C++  & sage  \\
    & Anna Sopena \\

>>>>>>> participants.tex
  \end{tabular}
\end{center}


\end{document}




<<<<<<< participants.tex
=======


>>>>>>> participants.tex
