\documentclass[10pt]{amsart}

\usepackage[latin1]{inputenc}
\usepackage{a4wide}
\usepackage{xcolor}
\usepackage{bbm}
\usepackage{hyperref}
\usepackage{paralist}

\newcommand{\RR}{\mathbbm{R}}

\begin{document}

\title{Contents and short bibliography\\Discrete and Algorithmic Geometry, UPC, 2019}
\author{Julian Pfeifle}
\maketitle

Due to my teaching (and grading) load this semester, I have not had time to prepare lecture notes for this class.
But since I more or less directly copied the content from various sources,
I hope to make your job of studying the material easier by explicitly listing the chapters I used.
A big thanks to Moritz Otth for pushing me do compile this list!

\bigskip
The overarching theme are realizations of oriented matroids.

\section{Matroids}

This material is directly copied from \cite[Lecture 1]{Reiner05}.

\renewcommand*{\descriptionlabel}[1]{\hspace\labelsep\normalfont\emph{#1}:}
\begin{description}
\item[Examples]
Vector and point configurations;
algebraic matroids;
transversal matroids;
graphical matroids

\item[Axiom systems]
Independent sets;
bases;
circuits;
cocircuits;
rank function;
flats;
hereditarily pure simplicial complexes;
universally shellable simplicial
complexes;
greedily optimizable independent set systems

\item[Operations]
  Direct sum; deletion; contraction


\item[Oriented Matroids]
 Axiom systems: Circuits, cocircuits  

\end{description}
% \nocite{Ziegler95,Matousek02,deloera-rambau-santos10,bjorner-etal99,Schrijver03,
%  Thomas06,Beck-Robins15,Bokowski06,Reiner05,RichterGebert11,CLOS15,Segers04,MillerSturmfels05}

I didn't end up introducing Coxeter matroids, or talk much about matroid base polytopes.
Also, Chirotopes weren't introduced until later.

For the direct sum, deletion and contraction I followed \cite[Lecture 2]{Reiner05}.
I did not use~\cite{Bjorner-Etal99} except for its Theorem~7.4.2, even though it's great as a reference.

The exercises in this section were taken from~\cite{Reiner05}, \cite{Schrijver03}, \cite{Bokowski06}, \cite{Ardila-Billey07}, \cite{Ziegler95}.


\section{Oriented matroid / Gale duality}

LinVal($A$);
LinDep($A$);
AffVal($A$);
AffDep($A$).
Radon's Lemma.
Dictionary of linear/affine Gale transform:
Faces of convex hull;
convex position.
Chirotopes.
Cyclic polytopes and neighborliness.
Asymptotic Upper Bound Theorem for simplicial polytopes.
Arrangements of real algebraic varieties and 
the Milnor-Thom-Oleinik-Petrovski theorem.

The oriented matroid / Gale duality construction follows \cite[Chapter~6]{Ziegler95},
as does the discussion of Radon partitions, (affine) Gale diagrams, cyclic polytopes and neighborliness.
The example of a non-rational polytope is~\cite[Example~6.21]{Ziegler95}.
The Asymptotic Upper Bound Theorem and the
presentation of the Milnor-Thom-Oleinik-Petrovski theorem are from~\cite[Chapter~5]{Matousek02},
which also has a good introduction to polytopes and Gale duality.

\section{Regular triangulations and the secondary polytope}

Regular polyhedral subdivisions from projections of lower faces.
The Union and Intersection properties.
The refinement partial order.
Examples.
The GKZ vector of a triangulation.
The secondary polytope and its vertices and affine hull.



I initially tried to follow \cite[Chapter~5]{deloera-rambau-santos10}, but found it to be too verbose for presentation in class.
For a leisurely introduction it works great, though.
In the end, I used~\cite[Chapters~7,8]{thomas06}.
An additional source is~\cite[Chapter~9]{Ziegler95}.

The exercises were taken from~\cite{deloera-rambau-santos10} and~\cite{Ziegler95}.

\section{Gr�bner bases}

Motivation: the Apollonius Circle Theorem.
Monomial/term/polynomial/support/ideal/variety.
Hilbert basis theorem.
Radical of an ideal.
Hilbert's Nullstellensatz.
Example: Algebraic attack on a small block cipher.
Monomial orders.
Division algorithm.
Elimination Theorem.

The primary sources here are~\cite[Chapter~2,~\S\S1--8 and Chapter~3,~\S1]{CLOS15} and~\cite[Chapters~10--12]{thomas06}.
A secondary source is~\cite{RichterGebert11}.
The example on the key sniffing attack is~\cite[Section~3.1]{Segers04}.

\section{The Grassmannian and flags}

Pl�cker coordinates of a point configuration.
The Grassmannian and Flag variety.
The matroid associated to a generic line arrangement via a scaffolding flag.
The initial monomials of the Pl�cker ideal as the incomparable pairs in the straightening poset.

The material on Pl�cker coordinates and the flag variety is from~\cite[Sections~14.1~and~14.2]{MillerSturmfels05}.
The matroid associated to a generic line arrangement (which also makes an appearance in the exercises) is due to~\cite{Ardila-Billey07}.


\section{Slack realization space of polytopes and oriented matroids}

Slack matrix, symbolic slack matrix, generalized slack matrix, and slack ideal of a polytope or matroid.
Ideal quotient and saturation with respect to a principal ideal.
Realization spaces of polytopes;
Mn\"ev's Universality Theorem.
Relationship between slack variety and generalized slack matrices.
Simplification of slack ideal via row and column rescaling.
Examples.

The relevant papers are \cite{Gouveia-etal19} and \cite{Brandt-wiebe19}.


\bibliographystyle{alpha}
\bibliography{bibliography} 

\end{document}
