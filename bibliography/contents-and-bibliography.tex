\documentclass[10pt]{amsart}

\usepackage[latin1]{inputenc}
\usepackage{a4wide}
\usepackage{xcolor}
\usepackage{bbm}
\usepackage{hyperref}

\newcommand{\RR}{\mathbbm{R}}

\begin{document}

\title{Contents and short bibliography\\Discrete and Algorithmic Geometry, UPC, 2019}
\author{Julian Pfeifle}
\maketitle

Due to my teaching (and grading) load this semester, I have not had time to prepare lecture notes for this class.
But since I more or less directly copied the content from various sources,
I hope to make your job of studying the material easier by explicitly listing the chapters I used.
A big thanks to Moritz Otth for pushing me do compile this list!

\bigskip
The overarching theme are realizations of oriented matroids.

\section{Examples and axiom systems}

This material is directly copied from \cite[Lecture 1]{Reiner05}.

\subsection{Matroids}

Examples:

\begin{itemize}
\item
  Vector and point configurations
\item
  algebraic matroids
\item
  transversal matroids
\item
  graphical matroids
\end{itemize}

Axiom systems:

\begin{itemize}
\item
  Independent sets
\item
  bases
\item
  circuits
\item
  cocircuits
\item
  rank function
\item
  flats
\item
  hereditarily pure simplicial complexes
\item
  universally shellable simplicial
  complexes
\item
  greedily optimizable independent set systems
\end{itemize}

I didn't end up introducing Coxeter matroids, or talk much about matroid base polytopes.

\subsection{Oriented Matroids}

%\nocite{Ziegler95,Matousek02,deloera-rambau-santos10,bjorner-etal99,Schrijver03,
%  Thomas06,Beck-Robins15,Bokowski06,Reiner05,RichterGebert11,CLOS15,Segers04,MillerSturmfels05}

Axiom systems:

\begin{itemize}
\item
  Circuits
\item
  Cocircuits
\end{itemize}

Chirotopes weren't introduced until later.
I did not use~\cite{Bjorner-Etal99} except for its Theorem~7.4.2, even though it's great as a reference.

The exercises in this section were taken from~\cite{Reiner05}, \cite{Schrijver03}, \cite{Bokowski06}, \cite{Ardila-Billey07}, \cite{Ziegler95}.

\section{Constructions}

After covering the direct sum, deletion and contraction following \cite[Lecture 2]{Reiner05}, we switch sources.

\section{Oriented matroid / Gale duality}

The oriented matroid / Gale duality construction follows \cite[Chapter~6]{Ziegler95},
as does the discussion of Radon partitions and affine Gale diagrams.
The example of a non-rational polytope is~\cite[Example~6.21]{Ziegler95}.
The presentation of the Milnor-Thom-Oleinik-Petrovski theorem is from~\cite{Matousek02}.

\subsection{Regular triangulations and the secondary polytope}

I initially tried to follow \cite[Chapter~5]{deloera-rambau-santos10}, but found it to be too verbose for presentation in class.
For a leisurely introduction it works great, though.
In the end, I used~\cite[Chapters~7,8]{thomas06}.
An additional source is~\cite[Chapter~9]{Ziegler95}.

The exercises were taken from~\cite{deloera-rambau-santos10} and~\cite{Ziegler95}.

\section{Gr�bner bases and the Grassmannian}

The primary sources here are~\cite[Chapter~2]{CLOS15} and~\cite[Chapters~10--12]{thomas06}.
A secondary source is~\cite{RichterGebert11}.
The example on the key sniffing attack is~\cite[Section~3.1]{Segers04}.
The Gr�bner basis of the Pl�cker ideal is from~\cite[Chapter~14]{MillerSturmfels05}.

\section{Realizations of oriented matroids}

The relevant papers are \cite{Gouveia-etal19} and \cite{Brandt-wiebe19}.


\bibliographystyle{plain}
\bibliography{bibliography}

\end{document}
